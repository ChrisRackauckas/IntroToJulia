
% Default to the notebook output style

    


% Inherit from the specified cell style.




    
\documentclass[11pt]{article}

    
    
    \usepackage[T1]{fontenc}
    % Nicer default font (+ math font) than Computer Modern for most use cases
    \usepackage{mathpazo}

    % Basic figure setup, for now with no caption control since it's done
    % automatically by Pandoc (which extracts ![](path) syntax from Markdown).
    \usepackage{graphicx}
    % We will generate all images so they have a width \maxwidth. This means
    % that they will get their normal width if they fit onto the page, but
    % are scaled down if they would overflow the margins.
    \makeatletter
    \def\maxwidth{\ifdim\Gin@nat@width>\linewidth\linewidth
    \else\Gin@nat@width\fi}
    \makeatother
    \let\Oldincludegraphics\includegraphics
    % Set max figure width to be 80% of text width, for now hardcoded.
    \renewcommand{\includegraphics}[1]{\Oldincludegraphics[width=.8\maxwidth]{#1}}
    % Ensure that by default, figures have no caption (until we provide a
    % proper Figure object with a Caption API and a way to capture that
    % in the conversion process - todo).
    \usepackage{caption}
    \DeclareCaptionLabelFormat{nolabel}{}
    \captionsetup{labelformat=nolabel}

    \usepackage{adjustbox} % Used to constrain images to a maximum size 
    \usepackage{xcolor} % Allow colors to be defined
    \usepackage{enumerate} % Needed for markdown enumerations to work
    \usepackage{geometry} % Used to adjust the document margins
    \usepackage{amsmath} % Equations
    \usepackage{amssymb} % Equations
    \usepackage{textcomp} % defines textquotesingle
    % Hack from http://tex.stackexchange.com/a/47451/13684:
    \AtBeginDocument{%
        \def\PYZsq{\textquotesingle}% Upright quotes in Pygmentized code
    }
    \usepackage{upquote} % Upright quotes for verbatim code
    \usepackage{eurosym} % defines \euro
    \usepackage[mathletters]{ucs} % Extended unicode (utf-8) support
    \usepackage[utf8x]{inputenc} % Allow utf-8 characters in the tex document
    \usepackage{fancyvrb} % verbatim replacement that allows latex
    \usepackage{grffile} % extends the file name processing of package graphics 
                         % to support a larger range 
    % The hyperref package gives us a pdf with properly built
    % internal navigation ('pdf bookmarks' for the table of contents,
    % internal cross-reference links, web links for URLs, etc.)
    \usepackage{hyperref}
    \usepackage{longtable} % longtable support required by pandoc >1.10
    \usepackage{booktabs}  % table support for pandoc > 1.12.2
    \usepackage[inline]{enumitem} % IRkernel/repr support (it uses the enumerate* environment)
    \usepackage[normalem]{ulem} % ulem is needed to support strikethroughs (\sout)
                                % normalem makes italics be italics, not underlines
    

    
    
    % Colors for the hyperref package
    \definecolor{urlcolor}{rgb}{0,.145,.698}
    \definecolor{linkcolor}{rgb}{.71,0.21,0.01}
    \definecolor{citecolor}{rgb}{.12,.54,.11}

    % ANSI colors
    \definecolor{ansi-black}{HTML}{3E424D}
    \definecolor{ansi-black-intense}{HTML}{282C36}
    \definecolor{ansi-red}{HTML}{E75C58}
    \definecolor{ansi-red-intense}{HTML}{B22B31}
    \definecolor{ansi-green}{HTML}{00A250}
    \definecolor{ansi-green-intense}{HTML}{007427}
    \definecolor{ansi-yellow}{HTML}{DDB62B}
    \definecolor{ansi-yellow-intense}{HTML}{B27D12}
    \definecolor{ansi-blue}{HTML}{208FFB}
    \definecolor{ansi-blue-intense}{HTML}{0065CA}
    \definecolor{ansi-magenta}{HTML}{D160C4}
    \definecolor{ansi-magenta-intense}{HTML}{A03196}
    \definecolor{ansi-cyan}{HTML}{60C6C8}
    \definecolor{ansi-cyan-intense}{HTML}{258F8F}
    \definecolor{ansi-white}{HTML}{C5C1B4}
    \definecolor{ansi-white-intense}{HTML}{A1A6B2}

    % commands and environments needed by pandoc snippets
    % extracted from the output of `pandoc -s`
    \providecommand{\tightlist}{%
      \setlength{\itemsep}{0pt}\setlength{\parskip}{0pt}}
    \DefineVerbatimEnvironment{Highlighting}{Verbatim}{commandchars=\\\{\}}
    % Add ',fontsize=\small' for more characters per line
    \newenvironment{Shaded}{}{}
    \newcommand{\KeywordTok}[1]{\textcolor[rgb]{0.00,0.44,0.13}{\textbf{{#1}}}}
    \newcommand{\DataTypeTok}[1]{\textcolor[rgb]{0.56,0.13,0.00}{{#1}}}
    \newcommand{\DecValTok}[1]{\textcolor[rgb]{0.25,0.63,0.44}{{#1}}}
    \newcommand{\BaseNTok}[1]{\textcolor[rgb]{0.25,0.63,0.44}{{#1}}}
    \newcommand{\FloatTok}[1]{\textcolor[rgb]{0.25,0.63,0.44}{{#1}}}
    \newcommand{\CharTok}[1]{\textcolor[rgb]{0.25,0.44,0.63}{{#1}}}
    \newcommand{\StringTok}[1]{\textcolor[rgb]{0.25,0.44,0.63}{{#1}}}
    \newcommand{\CommentTok}[1]{\textcolor[rgb]{0.38,0.63,0.69}{\textit{{#1}}}}
    \newcommand{\OtherTok}[1]{\textcolor[rgb]{0.00,0.44,0.13}{{#1}}}
    \newcommand{\AlertTok}[1]{\textcolor[rgb]{1.00,0.00,0.00}{\textbf{{#1}}}}
    \newcommand{\FunctionTok}[1]{\textcolor[rgb]{0.02,0.16,0.49}{{#1}}}
    \newcommand{\RegionMarkerTok}[1]{{#1}}
    \newcommand{\ErrorTok}[1]{\textcolor[rgb]{1.00,0.00,0.00}{\textbf{{#1}}}}
    \newcommand{\NormalTok}[1]{{#1}}
    
    % Additional commands for more recent versions of Pandoc
    \newcommand{\ConstantTok}[1]{\textcolor[rgb]{0.53,0.00,0.00}{{#1}}}
    \newcommand{\SpecialCharTok}[1]{\textcolor[rgb]{0.25,0.44,0.63}{{#1}}}
    \newcommand{\VerbatimStringTok}[1]{\textcolor[rgb]{0.25,0.44,0.63}{{#1}}}
    \newcommand{\SpecialStringTok}[1]{\textcolor[rgb]{0.73,0.40,0.53}{{#1}}}
    \newcommand{\ImportTok}[1]{{#1}}
    \newcommand{\DocumentationTok}[1]{\textcolor[rgb]{0.73,0.13,0.13}{\textit{{#1}}}}
    \newcommand{\AnnotationTok}[1]{\textcolor[rgb]{0.38,0.63,0.69}{\textbf{\textit{{#1}}}}}
    \newcommand{\CommentVarTok}[1]{\textcolor[rgb]{0.38,0.63,0.69}{\textbf{\textit{{#1}}}}}
    \newcommand{\VariableTok}[1]{\textcolor[rgb]{0.10,0.09,0.49}{{#1}}}
    \newcommand{\ControlFlowTok}[1]{\textcolor[rgb]{0.00,0.44,0.13}{\textbf{{#1}}}}
    \newcommand{\OperatorTok}[1]{\textcolor[rgb]{0.40,0.40,0.40}{{#1}}}
    \newcommand{\BuiltInTok}[1]{{#1}}
    \newcommand{\ExtensionTok}[1]{{#1}}
    \newcommand{\PreprocessorTok}[1]{\textcolor[rgb]{0.74,0.48,0.00}{{#1}}}
    \newcommand{\AttributeTok}[1]{\textcolor[rgb]{0.49,0.56,0.16}{{#1}}}
    \newcommand{\InformationTok}[1]{\textcolor[rgb]{0.38,0.63,0.69}{\textbf{\textit{{#1}}}}}
    \newcommand{\WarningTok}[1]{\textcolor[rgb]{0.38,0.63,0.69}{\textbf{\textit{{#1}}}}}
    
    
    % Define a nice break command that doesn't care if a line doesn't already
    % exist.
    \def\br{\hspace*{\fill} \\* }
    % Math Jax compatability definitions
    \def\gt{>}
    \def\lt{<}
    % Document parameters
    \title{ArraysAndMatrices}
    
    
    

    % Pygments definitions
    
\makeatletter
\def\PY@reset{\let\PY@it=\relax \let\PY@bf=\relax%
    \let\PY@ul=\relax \let\PY@tc=\relax%
    \let\PY@bc=\relax \let\PY@ff=\relax}
\def\PY@tok#1{\csname PY@tok@#1\endcsname}
\def\PY@toks#1+{\ifx\relax#1\empty\else%
    \PY@tok{#1}\expandafter\PY@toks\fi}
\def\PY@do#1{\PY@bc{\PY@tc{\PY@ul{%
    \PY@it{\PY@bf{\PY@ff{#1}}}}}}}
\def\PY#1#2{\PY@reset\PY@toks#1+\relax+\PY@do{#2}}

\expandafter\def\csname PY@tok@w\endcsname{\def\PY@tc##1{\textcolor[rgb]{0.73,0.73,0.73}{##1}}}
\expandafter\def\csname PY@tok@c\endcsname{\let\PY@it=\textit\def\PY@tc##1{\textcolor[rgb]{0.25,0.50,0.50}{##1}}}
\expandafter\def\csname PY@tok@cp\endcsname{\def\PY@tc##1{\textcolor[rgb]{0.74,0.48,0.00}{##1}}}
\expandafter\def\csname PY@tok@k\endcsname{\let\PY@bf=\textbf\def\PY@tc##1{\textcolor[rgb]{0.00,0.50,0.00}{##1}}}
\expandafter\def\csname PY@tok@kp\endcsname{\def\PY@tc##1{\textcolor[rgb]{0.00,0.50,0.00}{##1}}}
\expandafter\def\csname PY@tok@kt\endcsname{\def\PY@tc##1{\textcolor[rgb]{0.69,0.00,0.25}{##1}}}
\expandafter\def\csname PY@tok@o\endcsname{\def\PY@tc##1{\textcolor[rgb]{0.40,0.40,0.40}{##1}}}
\expandafter\def\csname PY@tok@ow\endcsname{\let\PY@bf=\textbf\def\PY@tc##1{\textcolor[rgb]{0.67,0.13,1.00}{##1}}}
\expandafter\def\csname PY@tok@nb\endcsname{\def\PY@tc##1{\textcolor[rgb]{0.00,0.50,0.00}{##1}}}
\expandafter\def\csname PY@tok@nf\endcsname{\def\PY@tc##1{\textcolor[rgb]{0.00,0.00,1.00}{##1}}}
\expandafter\def\csname PY@tok@nc\endcsname{\let\PY@bf=\textbf\def\PY@tc##1{\textcolor[rgb]{0.00,0.00,1.00}{##1}}}
\expandafter\def\csname PY@tok@nn\endcsname{\let\PY@bf=\textbf\def\PY@tc##1{\textcolor[rgb]{0.00,0.00,1.00}{##1}}}
\expandafter\def\csname PY@tok@ne\endcsname{\let\PY@bf=\textbf\def\PY@tc##1{\textcolor[rgb]{0.82,0.25,0.23}{##1}}}
\expandafter\def\csname PY@tok@nv\endcsname{\def\PY@tc##1{\textcolor[rgb]{0.10,0.09,0.49}{##1}}}
\expandafter\def\csname PY@tok@no\endcsname{\def\PY@tc##1{\textcolor[rgb]{0.53,0.00,0.00}{##1}}}
\expandafter\def\csname PY@tok@nl\endcsname{\def\PY@tc##1{\textcolor[rgb]{0.63,0.63,0.00}{##1}}}
\expandafter\def\csname PY@tok@ni\endcsname{\let\PY@bf=\textbf\def\PY@tc##1{\textcolor[rgb]{0.60,0.60,0.60}{##1}}}
\expandafter\def\csname PY@tok@na\endcsname{\def\PY@tc##1{\textcolor[rgb]{0.49,0.56,0.16}{##1}}}
\expandafter\def\csname PY@tok@nt\endcsname{\let\PY@bf=\textbf\def\PY@tc##1{\textcolor[rgb]{0.00,0.50,0.00}{##1}}}
\expandafter\def\csname PY@tok@nd\endcsname{\def\PY@tc##1{\textcolor[rgb]{0.67,0.13,1.00}{##1}}}
\expandafter\def\csname PY@tok@s\endcsname{\def\PY@tc##1{\textcolor[rgb]{0.73,0.13,0.13}{##1}}}
\expandafter\def\csname PY@tok@sd\endcsname{\let\PY@it=\textit\def\PY@tc##1{\textcolor[rgb]{0.73,0.13,0.13}{##1}}}
\expandafter\def\csname PY@tok@si\endcsname{\let\PY@bf=\textbf\def\PY@tc##1{\textcolor[rgb]{0.73,0.40,0.53}{##1}}}
\expandafter\def\csname PY@tok@se\endcsname{\let\PY@bf=\textbf\def\PY@tc##1{\textcolor[rgb]{0.73,0.40,0.13}{##1}}}
\expandafter\def\csname PY@tok@sr\endcsname{\def\PY@tc##1{\textcolor[rgb]{0.73,0.40,0.53}{##1}}}
\expandafter\def\csname PY@tok@ss\endcsname{\def\PY@tc##1{\textcolor[rgb]{0.10,0.09,0.49}{##1}}}
\expandafter\def\csname PY@tok@sx\endcsname{\def\PY@tc##1{\textcolor[rgb]{0.00,0.50,0.00}{##1}}}
\expandafter\def\csname PY@tok@m\endcsname{\def\PY@tc##1{\textcolor[rgb]{0.40,0.40,0.40}{##1}}}
\expandafter\def\csname PY@tok@gh\endcsname{\let\PY@bf=\textbf\def\PY@tc##1{\textcolor[rgb]{0.00,0.00,0.50}{##1}}}
\expandafter\def\csname PY@tok@gu\endcsname{\let\PY@bf=\textbf\def\PY@tc##1{\textcolor[rgb]{0.50,0.00,0.50}{##1}}}
\expandafter\def\csname PY@tok@gd\endcsname{\def\PY@tc##1{\textcolor[rgb]{0.63,0.00,0.00}{##1}}}
\expandafter\def\csname PY@tok@gi\endcsname{\def\PY@tc##1{\textcolor[rgb]{0.00,0.63,0.00}{##1}}}
\expandafter\def\csname PY@tok@gr\endcsname{\def\PY@tc##1{\textcolor[rgb]{1.00,0.00,0.00}{##1}}}
\expandafter\def\csname PY@tok@ge\endcsname{\let\PY@it=\textit}
\expandafter\def\csname PY@tok@gs\endcsname{\let\PY@bf=\textbf}
\expandafter\def\csname PY@tok@gp\endcsname{\let\PY@bf=\textbf\def\PY@tc##1{\textcolor[rgb]{0.00,0.00,0.50}{##1}}}
\expandafter\def\csname PY@tok@go\endcsname{\def\PY@tc##1{\textcolor[rgb]{0.53,0.53,0.53}{##1}}}
\expandafter\def\csname PY@tok@gt\endcsname{\def\PY@tc##1{\textcolor[rgb]{0.00,0.27,0.87}{##1}}}
\expandafter\def\csname PY@tok@err\endcsname{\def\PY@bc##1{\setlength{\fboxsep}{0pt}\fcolorbox[rgb]{1.00,0.00,0.00}{1,1,1}{\strut ##1}}}
\expandafter\def\csname PY@tok@kc\endcsname{\let\PY@bf=\textbf\def\PY@tc##1{\textcolor[rgb]{0.00,0.50,0.00}{##1}}}
\expandafter\def\csname PY@tok@kd\endcsname{\let\PY@bf=\textbf\def\PY@tc##1{\textcolor[rgb]{0.00,0.50,0.00}{##1}}}
\expandafter\def\csname PY@tok@kn\endcsname{\let\PY@bf=\textbf\def\PY@tc##1{\textcolor[rgb]{0.00,0.50,0.00}{##1}}}
\expandafter\def\csname PY@tok@kr\endcsname{\let\PY@bf=\textbf\def\PY@tc##1{\textcolor[rgb]{0.00,0.50,0.00}{##1}}}
\expandafter\def\csname PY@tok@bp\endcsname{\def\PY@tc##1{\textcolor[rgb]{0.00,0.50,0.00}{##1}}}
\expandafter\def\csname PY@tok@fm\endcsname{\def\PY@tc##1{\textcolor[rgb]{0.00,0.00,1.00}{##1}}}
\expandafter\def\csname PY@tok@vc\endcsname{\def\PY@tc##1{\textcolor[rgb]{0.10,0.09,0.49}{##1}}}
\expandafter\def\csname PY@tok@vg\endcsname{\def\PY@tc##1{\textcolor[rgb]{0.10,0.09,0.49}{##1}}}
\expandafter\def\csname PY@tok@vi\endcsname{\def\PY@tc##1{\textcolor[rgb]{0.10,0.09,0.49}{##1}}}
\expandafter\def\csname PY@tok@vm\endcsname{\def\PY@tc##1{\textcolor[rgb]{0.10,0.09,0.49}{##1}}}
\expandafter\def\csname PY@tok@sa\endcsname{\def\PY@tc##1{\textcolor[rgb]{0.73,0.13,0.13}{##1}}}
\expandafter\def\csname PY@tok@sb\endcsname{\def\PY@tc##1{\textcolor[rgb]{0.73,0.13,0.13}{##1}}}
\expandafter\def\csname PY@tok@sc\endcsname{\def\PY@tc##1{\textcolor[rgb]{0.73,0.13,0.13}{##1}}}
\expandafter\def\csname PY@tok@dl\endcsname{\def\PY@tc##1{\textcolor[rgb]{0.73,0.13,0.13}{##1}}}
\expandafter\def\csname PY@tok@s2\endcsname{\def\PY@tc##1{\textcolor[rgb]{0.73,0.13,0.13}{##1}}}
\expandafter\def\csname PY@tok@sh\endcsname{\def\PY@tc##1{\textcolor[rgb]{0.73,0.13,0.13}{##1}}}
\expandafter\def\csname PY@tok@s1\endcsname{\def\PY@tc##1{\textcolor[rgb]{0.73,0.13,0.13}{##1}}}
\expandafter\def\csname PY@tok@mb\endcsname{\def\PY@tc##1{\textcolor[rgb]{0.40,0.40,0.40}{##1}}}
\expandafter\def\csname PY@tok@mf\endcsname{\def\PY@tc##1{\textcolor[rgb]{0.40,0.40,0.40}{##1}}}
\expandafter\def\csname PY@tok@mh\endcsname{\def\PY@tc##1{\textcolor[rgb]{0.40,0.40,0.40}{##1}}}
\expandafter\def\csname PY@tok@mi\endcsname{\def\PY@tc##1{\textcolor[rgb]{0.40,0.40,0.40}{##1}}}
\expandafter\def\csname PY@tok@il\endcsname{\def\PY@tc##1{\textcolor[rgb]{0.40,0.40,0.40}{##1}}}
\expandafter\def\csname PY@tok@mo\endcsname{\def\PY@tc##1{\textcolor[rgb]{0.40,0.40,0.40}{##1}}}
\expandafter\def\csname PY@tok@ch\endcsname{\let\PY@it=\textit\def\PY@tc##1{\textcolor[rgb]{0.25,0.50,0.50}{##1}}}
\expandafter\def\csname PY@tok@cm\endcsname{\let\PY@it=\textit\def\PY@tc##1{\textcolor[rgb]{0.25,0.50,0.50}{##1}}}
\expandafter\def\csname PY@tok@cpf\endcsname{\let\PY@it=\textit\def\PY@tc##1{\textcolor[rgb]{0.25,0.50,0.50}{##1}}}
\expandafter\def\csname PY@tok@c1\endcsname{\let\PY@it=\textit\def\PY@tc##1{\textcolor[rgb]{0.25,0.50,0.50}{##1}}}
\expandafter\def\csname PY@tok@cs\endcsname{\let\PY@it=\textit\def\PY@tc##1{\textcolor[rgb]{0.25,0.50,0.50}{##1}}}

\def\PYZbs{\char`\\}
\def\PYZus{\char`\_}
\def\PYZob{\char`\{}
\def\PYZcb{\char`\}}
\def\PYZca{\char`\^}
\def\PYZam{\char`\&}
\def\PYZlt{\char`\<}
\def\PYZgt{\char`\>}
\def\PYZsh{\char`\#}
\def\PYZpc{\char`\%}
\def\PYZdl{\char`\$}
\def\PYZhy{\char`\-}
\def\PYZsq{\char`\'}
\def\PYZdq{\char`\"}
\def\PYZti{\char`\~}
% for compatibility with earlier versions
\def\PYZat{@}
\def\PYZlb{[}
\def\PYZrb{]}
\makeatother


    % Exact colors from NB
    \definecolor{incolor}{rgb}{0.0, 0.0, 0.5}
    \definecolor{outcolor}{rgb}{0.545, 0.0, 0.0}



    
    % Prevent overflowing lines due to hard-to-break entities
    \sloppy 
    % Setup hyperref package
    \hypersetup{
      breaklinks=true,  % so long urls are correctly broken across lines
      colorlinks=true,
      urlcolor=urlcolor,
      linkcolor=linkcolor,
      citecolor=citecolor,
      }
    % Slightly bigger margins than the latex defaults
    
    \geometry{verbose,tmargin=1in,bmargin=1in,lmargin=1in,rmargin=1in}
    
    

    \begin{document}
    
    
    \maketitle
    
    

    
    \section{More Details on Arrays and
Matrices}\label{more-details-on-arrays-and-matrices}

    \subsection{Array Operations}\label{array-operations}

    One key feature for doing math in Julia are the \texttt{broadcast} and
\texttt{map} operations. The \texttt{map} operation is like an R or
MATLAB \texttt{apply} which applies a function to each element of an
array. For example, we can apply the anonymous function
\texttt{(x)-\textgreater{}x\^{}2} to each element via:

    \begin{Verbatim}[commandchars=\\\{\}]
{\color{incolor}In [{\color{incolor}1}]:} \PY{n}{map}\PY{p}{(}\PY{p}{(}\PY{n}{x}\PY{p}{)}\PY{o}{\PYZhy{}}\PY{o}{\PYZgt{}}\PY{n}{x}\PY{o}{\PYZca{}}\PY{l+m+mi}{2}\PY{p}{,}\PY{l+m+mi}{1}\PY{o}{:}\PY{l+m+mi}{5}\PY{p}{)}
\end{Verbatim}


\begin{Verbatim}[commandchars=\\\{\}]
{\color{outcolor}Out[{\color{outcolor}1}]:} 5-element Array\{Int64,1\}:
          1
          4
          9
         16
         25
\end{Verbatim}
            
    The \texttt{broadcast} operation is similar except it is for "elements
which have a shape" and it will "broadcast the shaped objects to a
common size by expanding singleton dimensions". For example, the
following broadcast's \texttt{+} onto \texttt{A} and \texttt{B}:

    \begin{Verbatim}[commandchars=\\\{\}]
{\color{incolor}In [{\color{incolor}2}]:} \PY{n}{A} \PY{o}{=} \PY{l+m+mi}{1}\PY{o}{:}\PY{l+m+mi}{5} \PY{c}{\PYZsh{} Acts like a column vector, Julia is \PYZdq{}column\PYZhy{}major\PYZdq{} so columns come first}
        \PY{n}{B} \PY{o}{=} \PY{p}{[}\PY{l+m+mi}{1} \PY{l+m+mi}{2}
             \PY{l+m+mi}{3} \PY{l+m+mi}{4}
             \PY{l+m+mi}{5} \PY{l+m+mi}{6}
             \PY{l+m+mi}{7} \PY{l+m+mi}{8}
             \PY{l+m+mi}{9} \PY{l+m+mi}{10}\PY{p}{]}
        \PY{n}{broadcast}\PY{p}{(}\PY{o}{+}\PY{p}{,}\PY{n}{A}\PY{p}{,}\PY{n}{B}\PY{p}{)}
\end{Verbatim}


\begin{Verbatim}[commandchars=\\\{\}]
{\color{outcolor}Out[{\color{outcolor}2}]:} 5×2 Array\{Int64,2\}:
          2   3
          5   6
          8   9
         11  12
         14  15
\end{Verbatim}
            
    If \texttt{A} and \texttt{B} are the same size, then broadcasting is the
same as mapping.

    \subsubsection{In-Depth Julia 1: Special Broadcasting
Syntax}\label{in-depth-julia-1-special-broadcasting-syntax}

One major area (which is under a lot of active development) is the
specialized broadcasting syntax. The short summary is, putting a
\texttt{.} with a function or operator causes it to broadcast. For
example, we can broadcast any function with the syntax \texttt{f.(x)},
and broadcast operators by \texttt{.+} and related. For example:

    \begin{Verbatim}[commandchars=\\\{\}]
{\color{incolor}In [{\color{incolor}3}]:} \PY{n}{A} \PY{o}{=} \PY{l+m+mi}{1}\PY{o}{:}\PY{l+m+mi}{5}
        \PY{n}{B} \PY{o}{=} \PY{p}{[}\PY{l+m+mi}{2}\PY{p}{;}\PY{l+m+mi}{3}\PY{p}{;}\PY{l+m+mi}{4}\PY{p}{;}\PY{l+m+mi}{5}\PY{p}{;}\PY{l+m+mi}{6}\PY{p}{]}
        \PY{n}{A}\PY{o}{.*}\PY{n}{B}
\end{Verbatim}


\begin{Verbatim}[commandchars=\\\{\}]
{\color{outcolor}Out[{\color{outcolor}3}]:} 5-element Array\{Int64,1\}:
          2
          6
         12
         20
         30
\end{Verbatim}
            
    People coming from MATLAB might recognize this as "element-wise
multiplication". If this was a basic introduction to Julia, I'd say this
was element-wise multiplication and be done with it. However, this is
the non-trivial introduction. {[}Note: Some of this is not true right
now (v0.5) but is becoming true...{]}.

While it looks the same to the user, the implementation is very
different In MATLAB and earlier versions of Julia, \texttt{.*} was an
operator. In Julia's more refined world, we can explain this as
\texttt{.*\{T\textless{}:Number,N\}(x::Array\{T,N\},y::Array\{T,N\})}
being a function, and \texttt{A.*B} calling this function. However, if
\texttt{.*} is just a function, then

    \begin{Verbatim}[commandchars=\\\{\}]
{\color{incolor}In [{\color{incolor}4}]:} \PY{n}{C} \PY{o}{=} \PY{p}{[}\PY{l+m+mi}{3}\PY{p}{;}\PY{l+m+mi}{4}\PY{p}{;}\PY{l+m+mi}{5}\PY{p}{;}\PY{l+m+mi}{2}\PY{p}{;}\PY{l+m+mi}{1}\PY{p}{]}
        \PY{n}{A}\PY{o}{.*}\PY{n}{B}\PY{o}{.*}\PY{n}{C}
\end{Verbatim}


\begin{Verbatim}[commandchars=\\\{\}]
{\color{outcolor}Out[{\color{outcolor}4}]:} 5-element Array\{Int64,1\}:
          6
         24
         60
         40
         30
\end{Verbatim}
            
    the operation \texttt{A.*B.*C} actually expands into
\texttt{.*(A,.*(B,C))}. Let's think of how we'd implement \texttt{.*}.

\begin{center}\rule{0.5\linewidth}{\linethickness}\end{center}

\paragraph{Question 1}\label{question-1}

How would you implement \texttt{broadcast\_mult} as a function (not
using broadcast)? Don't peak below!

\begin{center}\rule{0.5\linewidth}{\linethickness}\end{center}

    \begin{Verbatim}[commandchars=\\\{\}]
{\color{incolor}In [{\color{incolor}5}]:} \PY{k}{function} \PY{n}{broadcast\PYZus{}mult}\PY{p}{(}\PY{n}{x}\PY{o}{::}\PY{k+kt}{Array}\PY{p}{\PYZob{}}\PY{n}{T}\PY{p}{,}\PY{n}{N}\PY{p}{\PYZcb{}}\PY{p}{,}\PY{n}{y}\PY{o}{::}\PY{k+kt}{Array}\PY{p}{\PYZob{}}\PY{n}{T}\PY{p}{,}\PY{n}{N}\PY{p}{\PYZcb{}}\PY{p}{)} \PY{n}{where} \PY{p}{\PYZob{}}\PY{n}{T}\PY{o}{\PYZlt{}:}\PY{k+kt}{Number}\PY{p}{,}\PY{n}{N}\PY{p}{\PYZcb{}}
            \PY{n}{output} \PY{o}{=} \PY{n}{similar}\PY{p}{(}\PY{n}{x}\PY{p}{)} \PY{c}{\PYZsh{} Makes an array of similar size and shape as x}
            \PY{k}{for} \PY{n}{i} \PY{k+kp}{in} \PY{n}{eachindex}\PY{p}{(}\PY{n}{x}\PY{p}{)} \PY{c}{\PYZsh{} Let the iterator choose the fast linear indexing for x}
                \PY{n}{output}\PY{p}{[}\PY{n}{i}\PY{p}{]} \PY{o}{=} \PY{n}{x}\PY{p}{[}\PY{n}{i}\PY{p}{]}\PY{o}{*}\PY{n}{y}\PY{p}{[}\PY{n}{i}\PY{p}{]}
            \PY{k}{end}
            \PY{n}{output}
        \PY{k}{end}
\end{Verbatim}


\begin{Verbatim}[commandchars=\\\{\}]
{\color{outcolor}Out[{\color{outcolor}5}]:} broadcast\_mult (generic function with 1 method)
\end{Verbatim}
            
    Notice that \texttt{broadcast\_mult} creates an array every time it is
called. Therefore a naive approach where \texttt{.*} is a function
creates two arrays in the call \texttt{A.*B.*C}. We saw earlier that
reducing memory allocations leads to vastly improved performance, so a
better implementation would be to do this all together as one loop:

    \begin{Verbatim}[commandchars=\\\{\}]
{\color{incolor}In [{\color{incolor}6}]:} \PY{k}{function} \PY{n}{broadcast\PYZus{}mult}\PY{p}{(}\PY{n}{x}\PY{o}{::}\PY{k+kt}{Array}\PY{p}{\PYZob{}}\PY{n}{T}\PY{p}{,}\PY{n}{N}\PY{p}{\PYZcb{}}\PY{p}{,}\PY{n}{y}\PY{o}{::}\PY{k+kt}{Array}\PY{p}{\PYZob{}}\PY{n}{T}\PY{p}{,}\PY{n}{N}\PY{p}{\PYZcb{}}\PY{p}{,}\PY{n}{z}\PY{o}{::}\PY{k+kt}{Array}\PY{p}{\PYZob{}}\PY{n}{T}\PY{p}{,}\PY{n}{N}\PY{p}{\PYZcb{}}\PY{p}{)} \PY{n}{where} \PY{p}{\PYZob{}}\PY{n}{T}\PY{o}{\PYZlt{}:}\PY{k+kt}{Number}\PY{p}{,}\PY{n}{N}\PY{p}{\PYZcb{}}
            \PY{n}{output} \PY{o}{=} \PY{n}{similar}\PY{p}{(}\PY{n}{x}\PY{p}{)} \PY{c}{\PYZsh{} Makes an array of similar size and shape as x}
            \PY{k}{for} \PY{n}{i} \PY{k+kp}{in} \PY{n}{eachindex}\PY{p}{(}\PY{n}{x}\PY{p}{)} \PY{c}{\PYZsh{} Let the iterator choose the fast linear indexing for x}
                \PY{n}{output}\PY{p}{[}\PY{n}{i}\PY{p}{]} \PY{o}{=} \PY{n}{x}\PY{p}{[}\PY{n}{i}\PY{p}{]}\PY{o}{*}\PY{n}{y}\PY{p}{[}\PY{n}{i}\PY{p}{]}\PY{o}{*}\PY{n}{z}\PY{p}{[}\PY{n}{i}\PY{p}{]}
            \PY{k}{end}
            \PY{n}{output}
        \PY{k}{end}
\end{Verbatim}


\begin{Verbatim}[commandchars=\\\{\}]
{\color{outcolor}Out[{\color{outcolor}6}]:} broadcast\_mult (generic function with 2 methods)
\end{Verbatim}
            
    (but notice this doesn't really work because now \texttt{.*} isn't a
binary operator and therefore the inline syntax won't work). This
optimization is known as "loop fusing". Julia does this by searching for
all of the broadcasts in a line and putting them together into one
broadcast statement during parsing time. Therefore, in Julia
\texttt{A.*B.*C} creates an anonymous function and broadcasts on it,
like

    \begin{Verbatim}[commandchars=\\\{\}]
{\color{incolor}In [{\color{incolor}7}]:} \PY{n}{broadcast}\PY{p}{(}\PY{p}{(}\PY{n}{x}\PY{p}{,}\PY{n}{y}\PY{p}{,}\PY{n}{z}\PY{p}{)}\PY{o}{\PYZhy{}}\PY{o}{\PYZgt{}}\PY{n}{x}\PY{o}{*}\PY{n}{y}\PY{o}{*}\PY{n}{z}\PY{p}{,}\PY{n}{A}\PY{p}{,}\PY{n}{B}\PY{p}{,}\PY{n}{C}\PY{p}{)}
\end{Verbatim}


\begin{Verbatim}[commandchars=\\\{\}]
{\color{outcolor}Out[{\color{outcolor}7}]:} 5-element Array\{Int64,1\}:
          6
         24
         60
         40
         30
\end{Verbatim}
            
    Notice that this is equivalent to our 1-loop solution. However, because
all array-based math uses this broadcasting syntax with a \texttt{.},
Julia can fuse the broadcasts on all sorts of mathematical expressions
on arrays:

    \begin{Verbatim}[commandchars=\\\{\}]
{\color{incolor}In [{\color{incolor}8}]:} \PY{n}{A}\PY{o}{.*}\PY{n}{B}\PY{o}{.*}\PY{n}{sin}\PY{o}{.}\PY{p}{(}\PY{n}{C}\PY{p}{)}
\end{Verbatim}


\begin{Verbatim}[commandchars=\\\{\}]
{\color{outcolor}Out[{\color{outcolor}8}]:} 5-element Array\{Float64,1\}:
           0.2822400161197344
          -4.540814971847569 
         -11.507091295957661 
          18.185948536513635 
          25.244129544236895 
\end{Verbatim}
            
    One last thing to note is that we can also broadcast \texttt{=}. This
would be the same thing is as the loop \texttt{A{[}i{]}\ =\ ...} and
thus requires the array \texttt{A} to already be define. Thus for
example, if we let

    \begin{Verbatim}[commandchars=\\\{\}]
{\color{incolor}In [{\color{incolor}9}]:} \PY{n}{D} \PY{o}{=} \PY{n}{similar}\PY{p}{(}\PY{n}{C}\PY{p}{)}
\end{Verbatim}


\begin{Verbatim}[commandchars=\\\{\}]
{\color{outcolor}Out[{\color{outcolor}9}]:} 5-element Array\{Int64,1\}:
         140629431143280
         140629431143312
         140629431143344
         140629431144592
         140629431144432
\end{Verbatim}
            
    then the operation (Using \texttt{@btime} in BenchmarkTools.jl to get an
accurate measurement)

    \begin{Verbatim}[commandchars=\\\{\}]
{\color{incolor}In [{\color{incolor}10}]:} \PY{k}{using} \PY{n}{BenchmarkTools}
         \PY{n+nd}{@btime} \PY{n}{D}\PY{o}{.=}\PY{n}{A}\PY{o}{.*}\PY{n}{B}\PY{o}{.*}\PY{n}{C}
\end{Verbatim}


    \begin{Verbatim}[commandchars=\\\{\}]
  542.553 ns (4 allocations: 96 bytes)

    \end{Verbatim}

\begin{Verbatim}[commandchars=\\\{\}]
{\color{outcolor}Out[{\color{outcolor}10}]:} 5-element Array\{Int64,1\}:
           6
          24
          60
          40
          30
\end{Verbatim}
            
    does not allocate any arrays. Reducing temporary array allocations is
one way Julia outperforms other scientific computing languages.

Summary: \texttt{.} makes operations element-wise, but in a very smart
way.

    \subsection{Vectors, Matrices, and Linear
Algebra}\label{vectors-matrices-and-linear-algebra}

Julia's linear algebra syntax follows MATLAB's to a large extent (it's
just about the only thing MATLAB got right!). We already saw this a
little bit by seeing Julia's array indexing syntax. For example, we can
get the first three elements by \texttt{1:3}:

    \begin{Verbatim}[commandchars=\\\{\}]
{\color{incolor}In [{\color{incolor}11}]:} \PY{n}{A} \PY{o}{=} \PY{n}{rand}\PY{p}{(}\PY{l+m+mi}{4}\PY{p}{,}\PY{l+m+mi}{4}\PY{p}{)} \PY{c}{\PYZsh{} Generate a 4x4 random matrix}
         \PY{n}{A}\PY{p}{[}\PY{l+m+mi}{1}\PY{o}{:}\PY{l+m+mi}{3}\PY{p}{,}\PY{l+m+mi}{1}\PY{o}{:}\PY{l+m+mi}{3}\PY{p}{]} \PY{c}{\PYZsh{} Take the top left 3\PYZhy{}3 matrix}
\end{Verbatim}


\begin{Verbatim}[commandchars=\\\{\}]
{\color{outcolor}Out[{\color{outcolor}11}]:} 3×3 Array\{Float64,2\}:
          0.387462  0.13318   0.355203
          0.565691  0.881769  0.243633
          0.281771  0.618909  0.281218
\end{Verbatim}
            
    Note that Julia is column-major, meaning that columns come first in both
indexing order and in the computer's internal representation.

    \subsubsection{In-Depth Julia 2: Views}\label{in-depth-julia-2-views}

Notice that \texttt{A{[}1:3,1:3{]}} returned an array. Where did this
array come from? Well, since there was no 3x3 array before,
\texttt{A{[}1:3,1:3{]}} created an array (i.e. it had to allocate
memory)

    \begin{Verbatim}[commandchars=\\\{\}]
{\color{incolor}In [{\color{incolor}12}]:} \PY{n+nd}{@time} \PY{n}{A}\PY{p}{[}\PY{l+m+mi}{1}\PY{o}{:}\PY{l+m+mi}{3}\PY{p}{,}\PY{l+m+mi}{1}\PY{o}{:}\PY{l+m+mi}{3}\PY{p}{]}
\end{Verbatim}


    \begin{Verbatim}[commandchars=\\\{\}]
  0.000006 seconds (7 allocations: 384 bytes)

    \end{Verbatim}

\begin{Verbatim}[commandchars=\\\{\}]
{\color{outcolor}Out[{\color{outcolor}12}]:} 3×3 Array\{Float64,2\}:
          0.387462  0.13318   0.355203
          0.565691  0.881769  0.243633
          0.281771  0.618909  0.281218
\end{Verbatim}
            
    Do you always have to allocate memory when making new arrays? We saw
before this wasn't the case when dealing with references. Recall the
example where modifying one array modified another:

    \begin{Verbatim}[commandchars=\\\{\}]
{\color{incolor}In [{\color{incolor}13}]:} \PY{n}{a} \PY{o}{=} \PY{p}{[}\PY{l+m+mi}{1}\PY{p}{;}\PY{l+m+mi}{3}\PY{p}{;}\PY{l+m+mi}{5}\PY{p}{]}
         \PY{n+nd}{@time} \PY{n}{b} \PY{o}{=} \PY{n}{a}
         \PY{n}{a}\PY{p}{[}\PY{l+m+mi}{2}\PY{p}{]} \PY{o}{=} \PY{l+m+mi}{10}
         \PY{n}{a}
         \PY{n+nd}{@time} \PY{n}{c} \PY{o}{=} \PY{n}{copy}\PY{p}{(}\PY{n}{a}\PY{p}{)}
\end{Verbatim}


    \begin{Verbatim}[commandchars=\\\{\}]
  0.000001 seconds (4 allocations: 192 bytes)
  0.001286 seconds (28 allocations: 1.828 KiB)

    \end{Verbatim}

\begin{Verbatim}[commandchars=\\\{\}]
{\color{outcolor}Out[{\color{outcolor}13}]:} 3-element Array\{Int64,1\}:
           1
          10
           5
\end{Verbatim}
            
    Notice that in the first case making \texttt{b} didn't allocate an
array: it just made an object with a pointer (an Int64), and had that
pointer point to the same array as \texttt{a}. To better understand this
behavior and exploit to for major performance gains, we need to make a
distinction. The array itself is the memory layout. For Julia arrays,
this is actually a C-pointer to a contiguous 1-dimensional slot of
memory. The \texttt{Array} type in Julia (and thus \texttt{Vector} and
\texttt{Matrix} which are type-alises for \texttt{Array\{T,1\}} and
\texttt{Array\{T,2\}} respectively) is a "view" to to that actual array.
A view is a type which points to an array, and has a compatibility layer
that changes how things like the indexing works. For example: if we
define the matrix

    \begin{Verbatim}[commandchars=\\\{\}]
{\color{incolor}In [{\color{incolor}14}]:} \PY{n}{A} \PY{o}{=} \PY{n}{rand}\PY{p}{(}\PY{l+m+mi}{4}\PY{p}{,}\PY{l+m+mi}{4}\PY{p}{)}
\end{Verbatim}


\begin{Verbatim}[commandchars=\\\{\}]
{\color{outcolor}Out[{\color{outcolor}14}]:} 4×4 Array\{Float64,2\}:
          0.854392   0.201331  0.437573  0.973475
          0.463982   0.690305  0.581859  0.630324
          0.370376   0.110284  0.893183  0.267128
          0.0583479  0.246125  0.944985  0.10474 
\end{Verbatim}
            
    then the array that we made is actually a 16-number long sequence (of
64-bit Floats) in memory, and \texttt{A} is a view to that array which
makes it index "like" it was 2-dimensional (reading down the columns).
This tells us one thing already: looping through the columns is faster
than looping through the rows. Indeed we can easily test this:

    \begin{Verbatim}[commandchars=\\\{\}]
{\color{incolor}In [{\color{incolor}15}]:} \PY{k}{function} \PY{n}{testloops}\PY{p}{(}\PY{p}{)}
             \PY{n}{b} \PY{o}{=} \PY{n}{rand}\PY{p}{(}\PY{l+m+mi}{1000}\PY{p}{,}\PY{l+m+mi}{1000}\PY{p}{)}
             \PY{n}{c} \PY{o}{=} \PY{l+m+mi}{0} \PY{c}{\PYZsh{} Need this so that way the compiler doesn\PYZsq{}t optimize away the loop!}
             \PY{n+nd}{@time} \PY{k}{for} \PY{n}{i} \PY{k+kp}{in} \PY{l+m+mi}{1}\PY{o}{:}\PY{l+m+mi}{1000}\PY{p}{,} \PY{n}{j} \PY{k+kp}{in} \PY{l+m+mi}{1}\PY{o}{:}\PY{l+m+mi}{1000}
                 \PY{n}{c}\PY{o}{+=}\PY{n}{b}\PY{p}{[}\PY{n}{i}\PY{p}{,}\PY{n}{j}\PY{p}{]}
             \PY{k}{end}
             \PY{n+nd}{@time} \PY{k}{for} \PY{n}{j} \PY{k+kp}{in} \PY{l+m+mi}{1}\PY{o}{:}\PY{l+m+mi}{1000}\PY{p}{,} \PY{n}{i} \PY{k+kp}{in} \PY{l+m+mi}{1}\PY{o}{:}\PY{l+m+mi}{1000}
                 \PY{n}{c}\PY{o}{+=}\PY{n}{b}\PY{p}{[}\PY{n}{i}\PY{p}{,}\PY{n}{j}\PY{p}{]}
             \PY{k}{end}
             \PY{n}{bidx} \PY{o}{=} \PY{n}{eachindex}\PY{p}{(}\PY{n}{b}\PY{p}{)}
             \PY{n+nd}{@time} \PY{k}{for} \PY{n}{i} \PY{k+kp}{in} \PY{n}{bidx}
                 \PY{n}{c}\PY{o}{+=}\PY{n}{b}\PY{p}{[}\PY{n}{i}\PY{p}{]}
             \PY{k}{end}
         \PY{k}{end}
         \PY{n}{testloops}\PY{p}{(}\PY{p}{)}
\end{Verbatim}


    \begin{Verbatim}[commandchars=\\\{\}]
  0.000839 seconds
  0.000281 seconds
  0.000287 seconds

    \end{Verbatim}

    One should normally use the \texttt{eachindex} function since this will
return the indices in the "fast" order for general iterator types.

In this terminology \texttt{A{[}1:3,1:3{]}} isn't a view to the same
memory. We can check this by noticing that it doesn't mutate the
original array:

    \begin{Verbatim}[commandchars=\\\{\}]
{\color{incolor}In [{\color{incolor}16}]:} \PY{n}{println}\PY{p}{(}\PY{n}{A}\PY{p}{)}
         \PY{n}{B} \PY{o}{=} \PY{n}{A}\PY{p}{[}\PY{l+m+mi}{1}\PY{o}{:}\PY{l+m+mi}{3}\PY{p}{,}\PY{l+m+mi}{1}\PY{o}{:}\PY{l+m+mi}{3}\PY{p}{]}
         \PY{n}{B}\PY{p}{[}\PY{l+m+mi}{1}\PY{p}{,}\PY{l+m+mi}{1}\PY{p}{]}\PY{o}{=}\PY{l+m+mi}{100}
         \PY{n}{println}\PY{p}{(}\PY{n}{A}\PY{p}{)}
\end{Verbatim}


    \begin{Verbatim}[commandchars=\\\{\}]
[0.854392 0.201331 0.437573 0.973475; 0.463982 0.690305 0.581859 0.630324; 0.370376 0.110284 0.893183 0.267128; 0.0583479 0.246125 0.944985 0.10474]
[0.854392 0.201331 0.437573 0.973475; 0.463982 0.690305 0.581859 0.630324; 0.370376 0.110284 0.893183 0.267128; 0.0583479 0.246125 0.944985 0.10474]

    \end{Verbatim}

    If we instead want a view, then we can use the \texttt{view} function:

    \begin{Verbatim}[commandchars=\\\{\}]
{\color{incolor}In [{\color{incolor}17}]:} \PY{n}{B} \PY{o}{=} \PY{n}{view}\PY{p}{(}\PY{n}{A}\PY{p}{,}\PY{l+m+mi}{1}\PY{o}{:}\PY{l+m+mi}{3}\PY{p}{,}\PY{l+m+mi}{1}\PY{o}{:}\PY{l+m+mi}{3}\PY{p}{)} \PY{c}{\PYZsh{} No copy involved}
         \PY{n}{B}\PY{p}{[}\PY{l+m+mi}{1}\PY{p}{,}\PY{l+m+mi}{1}\PY{p}{]} \PY{o}{=} \PY{l+m+mi}{100} \PY{c}{\PYZsh{} Will mutate A}
         \PY{n}{println}\PY{p}{(}\PY{n}{A}\PY{p}{)}
\end{Verbatim}


    \begin{Verbatim}[commandchars=\\\{\}]
[100.0 0.201331 0.437573 0.973475; 0.463982 0.690305 0.581859 0.630324; 0.370376 0.110284 0.893183 0.267128; 0.0583479 0.246125 0.944985 0.10474]

    \end{Verbatim}

    There are many cases where you might want to use a view. For example, if
a function needs the \texttt{i}th column, you may naively think of doing
\texttt{f(A{[}i,:{]})}. But, if \texttt{A} won't be changed in the loop,
we can avoid the memory allocation (and thus make things faster) by
sending a view to the original array which is simply the column:
\texttt{f(view(A,i,:))}. Two functions can be used to give common views.
\texttt{vec} gives a view of the array as a Vector and \texttt{reshape}
builds a view in a different shape. For example:

    \begin{Verbatim}[commandchars=\\\{\}]
{\color{incolor}In [{\color{incolor}18}]:} \PY{n}{C} \PY{o}{=} \PY{n}{vec}\PY{p}{(}\PY{n}{A}\PY{p}{)}
         \PY{n}{println}\PY{p}{(}\PY{n}{C}\PY{p}{)}
         \PY{n}{C} \PY{o}{=} \PY{n}{reshape}\PY{p}{(}\PY{n}{A}\PY{p}{,}\PY{l+m+mi}{8}\PY{p}{,}\PY{l+m+mi}{2}\PY{p}{)} \PY{c}{\PYZsh{} C is an 8x2 matrix}
         \PY{n}{C}
\end{Verbatim}


    \begin{Verbatim}[commandchars=\\\{\}]
[100.0, 0.463982, 0.370376, 0.0583479, 0.201331, 0.690305, 0.110284, 0.246125, 0.437573, 0.581859, 0.893183, 0.944985, 0.973475, 0.630324, 0.267128, 0.10474]

    \end{Verbatim}

\begin{Verbatim}[commandchars=\\\{\}]
{\color{outcolor}Out[{\color{outcolor}18}]:} 8×2 Array\{Float64,2\}:
          100.0        0.437573
            0.463982   0.581859
            0.370376   0.893183
            0.0583479  0.944985
            0.201331   0.973475
            0.690305   0.630324
            0.110284   0.267128
            0.246125   0.10474 
\end{Verbatim}
            
    Since these operations do not copy the array, they are very cheap and
can be used without worrying about performance issues.

    \subsubsection{Back to Linear Algebra}\label{back-to-linear-algebra}

Julia performs functions on matrices by default for dispatches on
matrices. For example, \texttt{+} is the matrix addition, while
\texttt{*} is matrix multiplication. Julia's \texttt{*} calls into a
program known as OpenBLAS so that way \texttt{*} is an optimized
multithreaded operation. For example:

    \begin{Verbatim}[commandchars=\\\{\}]
{\color{incolor}In [{\color{incolor}19}]:} \PY{n}{A} \PY{o}{=} \PY{n}{rand}\PY{p}{(}\PY{l+m+mi}{4}\PY{p}{,}\PY{l+m+mi}{4}\PY{p}{)}\PY{p}{;} \PY{n}{B} \PY{o}{=} \PY{n}{rand}\PY{p}{(}\PY{l+m+mi}{4}\PY{p}{,}\PY{l+m+mi}{4}\PY{p}{)}
         \PY{n}{C} \PY{o}{=} \PY{n}{A}\PY{o}{*}\PY{n}{B} \PY{c}{\PYZsh{} Matrix Multiplication}
         \PY{n}{D} \PY{o}{=} \PY{n}{A}\PY{o}{.*}\PY{n}{B} \PY{c}{\PYZsh{} Element\PYZhy{}wise Multiplication}
         \PY{n}{C}\PY{o}{\PYZhy{}}\PY{n}{D} \PY{c}{\PYZsh{} Not zero}
\end{Verbatim}


\begin{Verbatim}[commandchars=\\\{\}]
{\color{outcolor}Out[{\color{outcolor}19}]:} 4×4 Array\{Float64,2\}:
           1.01454     0.375748  0.701068  0.510709
           1.2425      0.390314  0.209259  0.605257
           1.01146    -0.354076  0.699519  1.03447 
          -0.0732094   0.322778  0.272802  0.359256
\end{Verbatim}
            
    A common operation is to solve the linear system \texttt{Ax=b}. In Julia
this is done by \texttt{A\textbackslash{}b}:

    \begin{Verbatim}[commandchars=\\\{\}]
{\color{incolor}In [{\color{incolor}20}]:} \PY{n}{b} \PY{o}{=} \PY{l+m+mi}{1}\PY{o}{:}\PY{l+m+mi}{4}
         \PY{n}{A}\PY{o}{\PYZbs{}}\PY{n}{b}
\end{Verbatim}


\begin{Verbatim}[commandchars=\\\{\}]
{\color{outcolor}Out[{\color{outcolor}20}]:} 4-element Array\{Float64,1\}:
           8.223195873659511 
           3.5244223769596212
           6.512869490976016 
          -5.568282770563026 
\end{Verbatim}
            
    Note that this uses a direct solver. Iterative solvers for linear
equations can be found in IterativeSolvers.jl hosted by the JuliaMath
organization.

    \subsection{A note about
"Vectorization"}\label{a-note-about-vectorization}

In MATLAB/Python/R you're told to "vectorize" your options, i.e. use
\texttt{apply} or these \texttt{.*} operations, in order to speed up
computations. This is because these operations call C programs which
will be faster than any interpreted MATLAB/Python/R loop. In Julia,
that's not the case: as long as your functions are type-stable, you will
get maximal performance. Thus vectorization does not improve
performance.

In fact, vectorization can reduce performance by creating "temporary
arrays". Those are the intermediate array allocations that come up with
doing operations like
\texttt{C{[}i,:{]}\ =\ A{[}i,:{]}\ .*\ B{[}i,:{]}}. In general, for the
best performance one should avoid vectorized calls or be careful to use
the broadcast/view syntax to define a statement without temporaries:

    \begin{Verbatim}[commandchars=\\\{\}]
{\color{incolor}In [{\color{incolor}21}]:} \PY{n}{i} \PY{o}{=} \PY{l+m+mi}{2}
         \PY{n}{C}\PY{p}{[}\PY{n}{i}\PY{p}{,}\PY{o}{:}\PY{p}{]} \PY{o}{.=} \PY{n}{view}\PY{p}{(}\PY{n}{A}\PY{p}{,}\PY{n}{i}\PY{p}{,}\PY{o}{:}\PY{p}{)} \PY{o}{.*} \PY{n}{view}\PY{p}{(}\PY{n}{B}\PY{p}{,}\PY{n}{i}\PY{p}{,}\PY{o}{:}\PY{p}{)}
\end{Verbatim}


\begin{Verbatim}[commandchars=\\\{\}]
{\color{outcolor}Out[{\color{outcolor}21}]:} 4-element view(::Array\{Float64,2\}, 2, :) with eltype Float64:
          0.00992595816308215 
          0.006518944651236117
          0.4009196761009102  
          0.6813548253594051  
\end{Verbatim}
            
    Note the odd quirk: array indexing is a view when on the left-hand side

\paragraph{Discussion: why is this the
case?}\label{discussion-why-is-this-the-case}

    \subsection{Sprase Matrices}\label{sprase-matrices}

Sprase Matrix capabilities are provided by SuiteSparse in the
SparseArrays standard library. Note that these are saved in a table
format, where there are triplets (i,j,value) which denote the existance
of a non-zero element at \texttt{(i,j)} of value \texttt{value}. A
sparse matrix can be created through the \texttt{sparse} command:

    \begin{Verbatim}[commandchars=\\\{\}]
{\color{incolor}In [{\color{incolor}22}]:} \PY{k}{using} \PY{n}{SparseArrays} \PY{c}{\PYZsh{} from the standard library, no installation required}
         \PY{n}{A} \PY{o}{=} \PY{n}{sparse}\PY{p}{(}\PY{p}{[}\PY{l+m+mi}{1}\PY{p}{;}\PY{l+m+mi}{2}\PY{p}{;}\PY{l+m+mi}{3}\PY{p}{]}\PY{p}{,}\PY{p}{[}\PY{l+m+mi}{2}\PY{p}{;}\PY{l+m+mi}{2}\PY{p}{;}\PY{l+m+mi}{1}\PY{p}{]}\PY{p}{,}\PY{p}{[}\PY{l+m+mi}{3}\PY{p}{;}\PY{l+m+mi}{4}\PY{p}{;}\PY{l+m+mi}{5}\PY{p}{]}\PY{p}{)}
\end{Verbatim}


\begin{Verbatim}[commandchars=\\\{\}]
{\color{outcolor}Out[{\color{outcolor}22}]:} 3×2 SparseMatrixCSC\{Int64,Int64\} with 3 stored entries:
           [3, 1]  =  5
           [1, 2]  =  3
           [2, 2]  =  4
\end{Verbatim}
            
    They can be converted into a dense matrix with the \texttt{Array}
command

    \begin{Verbatim}[commandchars=\\\{\}]
{\color{incolor}In [{\color{incolor}23}]:} \PY{k+kt}{Array}\PY{p}{(}\PY{n}{A}\PY{p}{)}
\end{Verbatim}


\begin{Verbatim}[commandchars=\\\{\}]
{\color{outcolor}Out[{\color{outcolor}23}]:} 3×2 Array\{Int64,2\}:
          0  3
          0  4
          5  0
\end{Verbatim}
            
    The documentation shows a lot more that you can do.

    \subsection{Special Matrix Types}\label{special-matrix-types}

    Like the rest of Julia, types and multiple dispatch is used to "secretly
enhance performance". There are many matrix types, so I will just show a
few and leave the rest to the documentation.

\subsubsection{Matrix Forms}\label{matrix-forms}

Many matrices follow specific forms: diagonal, tridiagonal, etc. Julia
has special types for these common matrix forms. For example, we can
define a \texttt{Tridiagonal} by giving it three vectors:

    \begin{Verbatim}[commandchars=\\\{\}]
{\color{incolor}In [{\color{incolor}24}]:} \PY{k}{using} \PY{n}{LinearAlgebra} \PY{c}{\PYZsh{} from the standard library, no install required}
         \PY{n}{A} \PY{o}{=} \PY{k+kt}{Tridiagonal}\PY{p}{(}\PY{l+m+mi}{2}\PY{o}{:}\PY{l+m+mi}{5}\PY{p}{,}\PY{l+m+mi}{1}\PY{o}{:}\PY{l+m+mi}{5}\PY{p}{,}\PY{l+m+mi}{1}\PY{o}{:}\PY{l+m+mi}{4}\PY{p}{)}
\end{Verbatim}


\begin{Verbatim}[commandchars=\\\{\}]
{\color{outcolor}Out[{\color{outcolor}24}]:} 5×5 Tridiagonal\{Int64,UnitRange\{Int64\}\}:
          1  1  ⋅  ⋅  ⋅
          2  2  2  ⋅  ⋅
          ⋅  3  3  3  ⋅
          ⋅  ⋅  4  4  4
          ⋅  ⋅  ⋅  5  5
\end{Verbatim}
            
    We can inspect it to see its internal form:

    \begin{Verbatim}[commandchars=\\\{\}]
{\color{incolor}In [{\color{incolor}25}]:} \PY{n}{fieldnames}\PY{p}{(}\PY{n}{typeof}\PY{p}{(}\PY{n}{A}\PY{p}{)}\PY{p}{)}
\end{Verbatim}


\begin{Verbatim}[commandchars=\\\{\}]
{\color{outcolor}Out[{\color{outcolor}25}]:} (:dl, :d, :du, :du2)
\end{Verbatim}
            
    \begin{Verbatim}[commandchars=\\\{\}]
{\color{incolor}In [{\color{incolor}26}]:} \PY{n}{A}\PY{o}{.}\PY{n}{d}
\end{Verbatim}


\begin{Verbatim}[commandchars=\\\{\}]
{\color{outcolor}Out[{\color{outcolor}26}]:} 1:5
\end{Verbatim}
            
    Notice that what the array stores is the vectors for the diagonals
themselves. It's clear to see that this gives a memory enhancement over
a dense matrix, and it gives a performance advantage because a dense
matrix would have an extra operation for each \texttt{0}. However, it's
also faster than a sprase matrix since a sparse matrix is stored as a
table \texttt{(i,j,value)} and retriving from the table has a bit of
overhead, while this is stored as 3 (actually 4...) contiguous arrays.
Therefore you get a performance boost by using a special matrix form
like \texttt{Tridiagonal} whenever one is available. Note that these
special matrix forms are outfitted with dispatches so that operations on
them work seamlessly like with normal matrices. For example, we can
multiply a Tridiagonal by a dense matrix:

    \begin{Verbatim}[commandchars=\\\{\}]
{\color{incolor}In [{\color{incolor}27}]:} \PY{n}{A}\PY{o}{*}\PY{n}{rand}\PY{p}{(}\PY{l+m+mi}{5}\PY{p}{,}\PY{l+m+mi}{5}\PY{p}{)}
\end{Verbatim}


\begin{Verbatim}[commandchars=\\\{\}]
{\color{outcolor}Out[{\color{outcolor}27}]:} 5×5 Array\{Float64,2\}:
          0.824599  1.88436  0.949077  1.01088  0.965401
          3.43658   4.1803   3.34217   2.12367  3.44722 
          5.49236   5.77771  6.49373   3.01978  7.17158 
          5.37533   4.84002  9.97901   2.68598  7.48032 
          2.2507    5.0211   8.86374   3.10273  5.55935 
\end{Verbatim}
            
    \subsubsection{The UniformScaling
Operator}\label{the-uniformscaling-operator}

One interesting type is the \texttt{UniformScaling} operator \texttt{I}.
Essentially, I uses dispatches to cleverly act like the identity matrix
without ever forming a matrix. For example, to mathematically subtract a
scalar \texttt{λ} from a matrix \texttt{A} we use the notation
\[ A - \lambda I\]

We can do this naturally with the \texttt{I} operator:

    \begin{Verbatim}[commandchars=\\\{\}]
{\color{incolor}In [{\color{incolor}28}]:} \PY{n}{λ} \PY{o}{=} \PY{l+m+mi}{2}
         \PY{n}{A} \PY{o}{=} \PY{n}{rand}\PY{p}{(}\PY{l+m+mi}{5}\PY{p}{,}\PY{l+m+mi}{5}\PY{p}{)}
         \PY{n}{display}\PY{p}{(}\PY{n}{A}\PY{p}{)}
         \PY{n}{A} \PY{o}{\PYZhy{}} \PY{n}{λ}\PY{o}{*}\PY{n+nb}{I}
\end{Verbatim}


    
    \begin{verbatim}
5×5 Array{Float64,2}:
 0.476364  0.469814   0.279262  0.818725  0.839773
 0.1453    0.475616   0.727647  0.791016  0.117332
 0.811776  0.0556756  0.639435  0.729035  0.770168
 0.202052  0.262302   0.562247  0.995452  0.711246
 0.647824  0.940714   0.535921  0.5515    0.988451
    \end{verbatim}

    
\begin{Verbatim}[commandchars=\\\{\}]
{\color{outcolor}Out[{\color{outcolor}28}]:} 5×5 Array\{Float64,2\}:
          -1.52364    0.469814    0.279262   0.818725   0.839773
           0.1453    -1.52438     0.727647   0.791016   0.117332
           0.811776   0.0556756  -1.36056    0.729035   0.770168
           0.202052   0.262302    0.562247  -1.00455    0.711246
           0.647824   0.940714    0.535921   0.5515    -1.01155 
\end{Verbatim}
            
    The MATLAB or NumPy way would be to create the identity matrix via a
command like \texttt{eye(5)}, but notice this prevents the allocation of
a 5x5 array. For large matrices, this operation is huge and thus this
could lead to some good performance improvements.

    \subsubsection{Factorization Forms}\label{factorization-forms}

One last type of interest are the factorization forms. In many cases,
you factorize a matrix using some factorization command in order to
speed up subsequence \texttt{A\textbackslash{}b} calls. Normally you
have to remember how this is done. For example, we can use a QR
factorization to solve a linear system like:

    \begin{Verbatim}[commandchars=\\\{\}]
{\color{incolor}In [{\color{incolor}29}]:} \PY{n}{b} \PY{o}{=} \PY{l+m+mi}{1}\PY{o}{:}\PY{l+m+mi}{5}
         \PY{n}{A} \PY{o}{=} \PY{n}{rand}\PY{p}{(}\PY{l+m+mi}{5}\PY{p}{,}\PY{l+m+mi}{5}\PY{p}{)}
         \PY{n}{Q}\PY{p}{,}\PY{n}{R} \PY{o}{=} \PY{n}{qr}\PY{p}{(}\PY{n}{A}\PY{p}{)}
         \PY{n}{println}\PY{p}{(}\PY{n}{A}\PY{o}{\PYZbs{}}\PY{n}{b}\PY{p}{)}
         \PY{n}{println}\PY{p}{(}\PY{n}{inv}\PY{p}{(}\PY{n}{R}\PY{p}{)}\PY{o}{*}\PY{n}{Q}\PY{o}{\PYZsq{}}\PY{o}{*}\PY{n}{b}\PY{p}{)}
\end{Verbatim}


    \begin{Verbatim}[commandchars=\\\{\}]
[2.59566, 4.60946, 3.79659, -4.36565, -0.832022]
[2.59566, 4.60946, 3.79659, -4.36565, -0.832022]

    \end{Verbatim}

    Thus we can save the variables \texttt{Q} and \texttt{R} and use
\texttt{inv(R)*Q\textquotesingle{}*b} instead of
\texttt{A\textbackslash{}b} and get better performance. This is the
NumPy/MATLAB way. However, that requires remembering the details of the
factorization. Instead, we can have Julia return a factorization type:

    \begin{Verbatim}[commandchars=\\\{\}]
{\color{incolor}In [{\color{incolor}30}]:} \PY{n}{q} \PY{o}{=} \PY{n}{qr}\PY{p}{(}\PY{n}{A}\PY{p}{)}
\end{Verbatim}


\begin{Verbatim}[commandchars=\\\{\}]
{\color{outcolor}Out[{\color{outcolor}30}]:} LinearAlgebra.QRCompactWY\{Float64,Array\{Float64,2\}\}
         Q factor:
         5×5 LinearAlgebra.QRCompactWYQ\{Float64,Array\{Float64,2\}\}:
          -0.255204    0.0591112  -0.793036    0.461912    0.29851  
          -0.337281   -0.114899   -0.254293   -0.0274575  -0.898677 
          -0.0272034   0.883421   -0.187797   -0.426877   -0.0365569
          -0.640822    0.318025    0.517295    0.468059    0.039169 
          -0.640094   -0.318955   -0.0597274  -0.620145    0.31686  
         R factor:
         5×5 Array\{Float64,2\}:
          -1.48729  -0.630203  -1.17066   -0.894171  -0.638356 
           0.0       0.727183   0.346293   0.581096  -0.0325812
           0.0       0.0       -0.337679  -0.176147  -0.503067 
           0.0       0.0        0.0        0.446083   0.185906 
           0.0       0.0        0.0        0.0       -0.159206 
\end{Verbatim}
            
    What this does is it internally stores \texttt{Qt}
(\texttt{Qt\ =\ Q\textquotesingle{}}) and \texttt{Rinv}
(\texttt{Rinv\ =\ inv(R)}). There is a dispatch defined for this type on
~which makes the \texttt{QRCompactWY} type perform the fast algorithm
\texttt{Rinv*Qt*b}, giving you the performance without having to
remember anything:

    \begin{Verbatim}[commandchars=\\\{\}]
{\color{incolor}In [{\color{incolor}31}]:} \PY{n}{q}\PY{o}{\PYZbs{}}\PY{n}{b}
\end{Verbatim}


\begin{Verbatim}[commandchars=\\\{\}]
{\color{outcolor}Out[{\color{outcolor}31}]:} 5-element Array\{Float64,1\}:
           2.595657162607796 
           4.609460214631678 
           3.7965882930591857
          -4.3656495522506535
          -0.8320217248698547
\end{Verbatim}
            
    The result is fast algorithms with clean code.

    \subsection{Random Numbers}\label{random-numbers}

One last little detail is for random numbers. Uniform random numbers are
generated by the \texttt{rand} function, while normal random numbers are
generated by the \texttt{randn} function.

    \begin{Verbatim}[commandchars=\\\{\}]
{\color{incolor}In [{\color{incolor}32}]:} \PY{n}{rand}\PY{p}{(}\PY{l+m+mi}{5}\PY{p}{)}
\end{Verbatim}


\begin{Verbatim}[commandchars=\\\{\}]
{\color{outcolor}Out[{\color{outcolor}32}]:} 5-element Array\{Float64,1\}:
          0.37140995537373   
          0.9442228213799317 
          0.10136688108233693
          0.15088132142777   
          0.34463385676227265
\end{Verbatim}
            
    \begin{Verbatim}[commandchars=\\\{\}]
{\color{incolor}In [{\color{incolor}33}]:} \PY{n}{randn}\PY{p}{(}\PY{l+m+mi}{5}\PY{p}{,}\PY{l+m+mi}{5}\PY{p}{)}
\end{Verbatim}


\begin{Verbatim}[commandchars=\\\{\}]
{\color{outcolor}Out[{\color{outcolor}33}]:} 5×5 Array\{Float64,2\}:
           0.201459    1.45915    -0.11018    0.516898  -0.563854
           2.369      -0.365038   -1.98576   -1.2159     0.182033
          -0.645141    0.59947     0.440944  -0.249403   1.41426 
           0.521324    0.0435694   1.07846   -1.08035   -0.344328
          -0.0253791   1.20466     1.0353    -1.93638    0.201323
\end{Verbatim}
            
    The argument is the size of the array. You can make random numbers which
match another array with the \texttt{size} function:

    \begin{Verbatim}[commandchars=\\\{\}]
{\color{incolor}In [{\color{incolor}34}]:} \PY{n}{a} \PY{o}{=} \PY{p}{[}\PY{l+m+mi}{1} \PY{l+m+mi}{2}
              \PY{l+m+mi}{3} \PY{l+m+mi}{4}\PY{p}{]}
         \PY{n}{randn}\PY{p}{(}\PY{n}{size}\PY{p}{(}\PY{n}{a}\PY{p}{)}\PY{p}{)}
\end{Verbatim}


\begin{Verbatim}[commandchars=\\\{\}]
{\color{outcolor}Out[{\color{outcolor}34}]:} 2×2 Array\{Float64,2\}:
          -0.952776  -0.78011
          -0.953679  -1.06923
\end{Verbatim}
            

    % Add a bibliography block to the postdoc
    
    
    
    \end{document}
