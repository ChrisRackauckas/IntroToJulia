
% Default to the notebook output style

    


% Inherit from the specified cell style.




    
\documentclass[11pt]{article}

    
    
    \usepackage[T1]{fontenc}
    % Nicer default font than Computer Modern for most use cases
    \usepackage{palatino}

    % Basic figure setup, for now with no caption control since it's done
    % automatically by Pandoc (which extracts ![](path) syntax from Markdown).
    \usepackage{graphicx}
    % We will generate all images so they have a width \maxwidth. This means
    % that they will get their normal width if they fit onto the page, but
    % are scaled down if they would overflow the margins.
    \makeatletter
    \def\maxwidth{\ifdim\Gin@nat@width>\linewidth\linewidth
    \else\Gin@nat@width\fi}
    \makeatother
    \let\Oldincludegraphics\includegraphics
    % Set max figure width to be 80% of text width, for now hardcoded.
    \renewcommand{\includegraphics}[1]{\Oldincludegraphics[width=.8\maxwidth]{#1}}
    % Ensure that by default, figures have no caption (until we provide a
    % proper Figure object with a Caption API and a way to capture that
    % in the conversion process - todo).
    \usepackage{caption}
    \DeclareCaptionLabelFormat{nolabel}{}
    \captionsetup{labelformat=nolabel}

    \usepackage{adjustbox} % Used to constrain images to a maximum size 
    \usepackage{xcolor} % Allow colors to be defined
    \usepackage{enumerate} % Needed for markdown enumerations to work
    \usepackage{geometry} % Used to adjust the document margins
    \usepackage{amsmath} % Equations
    \usepackage{amssymb} % Equations
    \usepackage{textcomp} % defines textquotesingle
    % Hack from http://tex.stackexchange.com/a/47451/13684:
    \AtBeginDocument{%
        \def\PYZsq{\textquotesingle}% Upright quotes in Pygmentized code
    }
    \usepackage{upquote} % Upright quotes for verbatim code
    \usepackage{eurosym} % defines \euro
    \usepackage[mathletters]{ucs} % Extended unicode (utf-8) support
    \usepackage[utf8x]{inputenc} % Allow utf-8 characters in the tex document
    \usepackage{fancyvrb} % verbatim replacement that allows latex
    \usepackage{grffile} % extends the file name processing of package graphics 
                         % to support a larger range 
    % The hyperref package gives us a pdf with properly built
    % internal navigation ('pdf bookmarks' for the table of contents,
    % internal cross-reference links, web links for URLs, etc.)
    \usepackage{hyperref}
    \usepackage{longtable} % longtable support required by pandoc >1.10
    \usepackage{booktabs}  % table support for pandoc > 1.12.2
    \usepackage[normalem]{ulem} % ulem is needed to support strikethroughs (\sout)
                                % normalem makes italics be italics, not underlines
    

    
    
    % Colors for the hyperref package
    \definecolor{urlcolor}{rgb}{0,.145,.698}
    \definecolor{linkcolor}{rgb}{.71,0.21,0.01}
    \definecolor{citecolor}{rgb}{.12,.54,.11}

    % ANSI colors
    \definecolor{ansi-black}{HTML}{3E424D}
    \definecolor{ansi-black-intense}{HTML}{282C36}
    \definecolor{ansi-red}{HTML}{E75C58}
    \definecolor{ansi-red-intense}{HTML}{B22B31}
    \definecolor{ansi-green}{HTML}{00A250}
    \definecolor{ansi-green-intense}{HTML}{007427}
    \definecolor{ansi-yellow}{HTML}{DDB62B}
    \definecolor{ansi-yellow-intense}{HTML}{B27D12}
    \definecolor{ansi-blue}{HTML}{208FFB}
    \definecolor{ansi-blue-intense}{HTML}{0065CA}
    \definecolor{ansi-magenta}{HTML}{D160C4}
    \definecolor{ansi-magenta-intense}{HTML}{A03196}
    \definecolor{ansi-cyan}{HTML}{60C6C8}
    \definecolor{ansi-cyan-intense}{HTML}{258F8F}
    \definecolor{ansi-white}{HTML}{C5C1B4}
    \definecolor{ansi-white-intense}{HTML}{A1A6B2}

    % commands and environments needed by pandoc snippets
    % extracted from the output of `pandoc -s`
    \providecommand{\tightlist}{%
      \setlength{\itemsep}{0pt}\setlength{\parskip}{0pt}}
    \DefineVerbatimEnvironment{Highlighting}{Verbatim}{commandchars=\\\{\}}
    % Add ',fontsize=\small' for more characters per line
    \newenvironment{Shaded}{}{}
    \newcommand{\KeywordTok}[1]{\textcolor[rgb]{0.00,0.44,0.13}{\textbf{{#1}}}}
    \newcommand{\DataTypeTok}[1]{\textcolor[rgb]{0.56,0.13,0.00}{{#1}}}
    \newcommand{\DecValTok}[1]{\textcolor[rgb]{0.25,0.63,0.44}{{#1}}}
    \newcommand{\BaseNTok}[1]{\textcolor[rgb]{0.25,0.63,0.44}{{#1}}}
    \newcommand{\FloatTok}[1]{\textcolor[rgb]{0.25,0.63,0.44}{{#1}}}
    \newcommand{\CharTok}[1]{\textcolor[rgb]{0.25,0.44,0.63}{{#1}}}
    \newcommand{\StringTok}[1]{\textcolor[rgb]{0.25,0.44,0.63}{{#1}}}
    \newcommand{\CommentTok}[1]{\textcolor[rgb]{0.38,0.63,0.69}{\textit{{#1}}}}
    \newcommand{\OtherTok}[1]{\textcolor[rgb]{0.00,0.44,0.13}{{#1}}}
    \newcommand{\AlertTok}[1]{\textcolor[rgb]{1.00,0.00,0.00}{\textbf{{#1}}}}
    \newcommand{\FunctionTok}[1]{\textcolor[rgb]{0.02,0.16,0.49}{{#1}}}
    \newcommand{\RegionMarkerTok}[1]{{#1}}
    \newcommand{\ErrorTok}[1]{\textcolor[rgb]{1.00,0.00,0.00}{\textbf{{#1}}}}
    \newcommand{\NormalTok}[1]{{#1}}
    
    % Additional commands for more recent versions of Pandoc
    \newcommand{\ConstantTok}[1]{\textcolor[rgb]{0.53,0.00,0.00}{{#1}}}
    \newcommand{\SpecialCharTok}[1]{\textcolor[rgb]{0.25,0.44,0.63}{{#1}}}
    \newcommand{\VerbatimStringTok}[1]{\textcolor[rgb]{0.25,0.44,0.63}{{#1}}}
    \newcommand{\SpecialStringTok}[1]{\textcolor[rgb]{0.73,0.40,0.53}{{#1}}}
    \newcommand{\ImportTok}[1]{{#1}}
    \newcommand{\DocumentationTok}[1]{\textcolor[rgb]{0.73,0.13,0.13}{\textit{{#1}}}}
    \newcommand{\AnnotationTok}[1]{\textcolor[rgb]{0.38,0.63,0.69}{\textbf{\textit{{#1}}}}}
    \newcommand{\CommentVarTok}[1]{\textcolor[rgb]{0.38,0.63,0.69}{\textbf{\textit{{#1}}}}}
    \newcommand{\VariableTok}[1]{\textcolor[rgb]{0.10,0.09,0.49}{{#1}}}
    \newcommand{\ControlFlowTok}[1]{\textcolor[rgb]{0.00,0.44,0.13}{\textbf{{#1}}}}
    \newcommand{\OperatorTok}[1]{\textcolor[rgb]{0.40,0.40,0.40}{{#1}}}
    \newcommand{\BuiltInTok}[1]{{#1}}
    \newcommand{\ExtensionTok}[1]{{#1}}
    \newcommand{\PreprocessorTok}[1]{\textcolor[rgb]{0.74,0.48,0.00}{{#1}}}
    \newcommand{\AttributeTok}[1]{\textcolor[rgb]{0.49,0.56,0.16}{{#1}}}
    \newcommand{\InformationTok}[1]{\textcolor[rgb]{0.38,0.63,0.69}{\textbf{\textit{{#1}}}}}
    \newcommand{\WarningTok}[1]{\textcolor[rgb]{0.38,0.63,0.69}{\textbf{\textit{{#1}}}}}
    
    
    % Define a nice break command that doesn't care if a line doesn't already
    % exist.
    \def\br{\hspace*{\fill} \\* }
    % Math Jax compatability definitions
    \def\gt{>}
    \def\lt{<}
    % Document parameters
    \title{JuliaMentalModel}
    
    
    

    % Pygments definitions
    
\makeatletter
\def\PY@reset{\let\PY@it=\relax \let\PY@bf=\relax%
    \let\PY@ul=\relax \let\PY@tc=\relax%
    \let\PY@bc=\relax \let\PY@ff=\relax}
\def\PY@tok#1{\csname PY@tok@#1\endcsname}
\def\PY@toks#1+{\ifx\relax#1\empty\else%
    \PY@tok{#1}\expandafter\PY@toks\fi}
\def\PY@do#1{\PY@bc{\PY@tc{\PY@ul{%
    \PY@it{\PY@bf{\PY@ff{#1}}}}}}}
\def\PY#1#2{\PY@reset\PY@toks#1+\relax+\PY@do{#2}}

\expandafter\def\csname PY@tok@gd\endcsname{\def\PY@tc##1{\textcolor[rgb]{0.63,0.00,0.00}{##1}}}
\expandafter\def\csname PY@tok@gu\endcsname{\let\PY@bf=\textbf\def\PY@tc##1{\textcolor[rgb]{0.50,0.00,0.50}{##1}}}
\expandafter\def\csname PY@tok@gt\endcsname{\def\PY@tc##1{\textcolor[rgb]{0.00,0.27,0.87}{##1}}}
\expandafter\def\csname PY@tok@gs\endcsname{\let\PY@bf=\textbf}
\expandafter\def\csname PY@tok@gr\endcsname{\def\PY@tc##1{\textcolor[rgb]{1.00,0.00,0.00}{##1}}}
\expandafter\def\csname PY@tok@cm\endcsname{\let\PY@it=\textit\def\PY@tc##1{\textcolor[rgb]{0.25,0.50,0.50}{##1}}}
\expandafter\def\csname PY@tok@vg\endcsname{\def\PY@tc##1{\textcolor[rgb]{0.10,0.09,0.49}{##1}}}
\expandafter\def\csname PY@tok@vi\endcsname{\def\PY@tc##1{\textcolor[rgb]{0.10,0.09,0.49}{##1}}}
\expandafter\def\csname PY@tok@vm\endcsname{\def\PY@tc##1{\textcolor[rgb]{0.10,0.09,0.49}{##1}}}
\expandafter\def\csname PY@tok@mh\endcsname{\def\PY@tc##1{\textcolor[rgb]{0.40,0.40,0.40}{##1}}}
\expandafter\def\csname PY@tok@cs\endcsname{\let\PY@it=\textit\def\PY@tc##1{\textcolor[rgb]{0.25,0.50,0.50}{##1}}}
\expandafter\def\csname PY@tok@ge\endcsname{\let\PY@it=\textit}
\expandafter\def\csname PY@tok@vc\endcsname{\def\PY@tc##1{\textcolor[rgb]{0.10,0.09,0.49}{##1}}}
\expandafter\def\csname PY@tok@il\endcsname{\def\PY@tc##1{\textcolor[rgb]{0.40,0.40,0.40}{##1}}}
\expandafter\def\csname PY@tok@go\endcsname{\def\PY@tc##1{\textcolor[rgb]{0.53,0.53,0.53}{##1}}}
\expandafter\def\csname PY@tok@cp\endcsname{\def\PY@tc##1{\textcolor[rgb]{0.74,0.48,0.00}{##1}}}
\expandafter\def\csname PY@tok@gi\endcsname{\def\PY@tc##1{\textcolor[rgb]{0.00,0.63,0.00}{##1}}}
\expandafter\def\csname PY@tok@gh\endcsname{\let\PY@bf=\textbf\def\PY@tc##1{\textcolor[rgb]{0.00,0.00,0.50}{##1}}}
\expandafter\def\csname PY@tok@ni\endcsname{\let\PY@bf=\textbf\def\PY@tc##1{\textcolor[rgb]{0.60,0.60,0.60}{##1}}}
\expandafter\def\csname PY@tok@nl\endcsname{\def\PY@tc##1{\textcolor[rgb]{0.63,0.63,0.00}{##1}}}
\expandafter\def\csname PY@tok@nn\endcsname{\let\PY@bf=\textbf\def\PY@tc##1{\textcolor[rgb]{0.00,0.00,1.00}{##1}}}
\expandafter\def\csname PY@tok@no\endcsname{\def\PY@tc##1{\textcolor[rgb]{0.53,0.00,0.00}{##1}}}
\expandafter\def\csname PY@tok@na\endcsname{\def\PY@tc##1{\textcolor[rgb]{0.49,0.56,0.16}{##1}}}
\expandafter\def\csname PY@tok@nb\endcsname{\def\PY@tc##1{\textcolor[rgb]{0.00,0.50,0.00}{##1}}}
\expandafter\def\csname PY@tok@nc\endcsname{\let\PY@bf=\textbf\def\PY@tc##1{\textcolor[rgb]{0.00,0.00,1.00}{##1}}}
\expandafter\def\csname PY@tok@nd\endcsname{\def\PY@tc##1{\textcolor[rgb]{0.67,0.13,1.00}{##1}}}
\expandafter\def\csname PY@tok@ne\endcsname{\let\PY@bf=\textbf\def\PY@tc##1{\textcolor[rgb]{0.82,0.25,0.23}{##1}}}
\expandafter\def\csname PY@tok@nf\endcsname{\def\PY@tc##1{\textcolor[rgb]{0.00,0.00,1.00}{##1}}}
\expandafter\def\csname PY@tok@si\endcsname{\let\PY@bf=\textbf\def\PY@tc##1{\textcolor[rgb]{0.73,0.40,0.53}{##1}}}
\expandafter\def\csname PY@tok@s2\endcsname{\def\PY@tc##1{\textcolor[rgb]{0.73,0.13,0.13}{##1}}}
\expandafter\def\csname PY@tok@nt\endcsname{\let\PY@bf=\textbf\def\PY@tc##1{\textcolor[rgb]{0.00,0.50,0.00}{##1}}}
\expandafter\def\csname PY@tok@nv\endcsname{\def\PY@tc##1{\textcolor[rgb]{0.10,0.09,0.49}{##1}}}
\expandafter\def\csname PY@tok@s1\endcsname{\def\PY@tc##1{\textcolor[rgb]{0.73,0.13,0.13}{##1}}}
\expandafter\def\csname PY@tok@dl\endcsname{\def\PY@tc##1{\textcolor[rgb]{0.73,0.13,0.13}{##1}}}
\expandafter\def\csname PY@tok@ch\endcsname{\let\PY@it=\textit\def\PY@tc##1{\textcolor[rgb]{0.25,0.50,0.50}{##1}}}
\expandafter\def\csname PY@tok@m\endcsname{\def\PY@tc##1{\textcolor[rgb]{0.40,0.40,0.40}{##1}}}
\expandafter\def\csname PY@tok@gp\endcsname{\let\PY@bf=\textbf\def\PY@tc##1{\textcolor[rgb]{0.00,0.00,0.50}{##1}}}
\expandafter\def\csname PY@tok@sh\endcsname{\def\PY@tc##1{\textcolor[rgb]{0.73,0.13,0.13}{##1}}}
\expandafter\def\csname PY@tok@ow\endcsname{\let\PY@bf=\textbf\def\PY@tc##1{\textcolor[rgb]{0.67,0.13,1.00}{##1}}}
\expandafter\def\csname PY@tok@sx\endcsname{\def\PY@tc##1{\textcolor[rgb]{0.00,0.50,0.00}{##1}}}
\expandafter\def\csname PY@tok@bp\endcsname{\def\PY@tc##1{\textcolor[rgb]{0.00,0.50,0.00}{##1}}}
\expandafter\def\csname PY@tok@c1\endcsname{\let\PY@it=\textit\def\PY@tc##1{\textcolor[rgb]{0.25,0.50,0.50}{##1}}}
\expandafter\def\csname PY@tok@fm\endcsname{\def\PY@tc##1{\textcolor[rgb]{0.00,0.00,1.00}{##1}}}
\expandafter\def\csname PY@tok@o\endcsname{\def\PY@tc##1{\textcolor[rgb]{0.40,0.40,0.40}{##1}}}
\expandafter\def\csname PY@tok@kc\endcsname{\let\PY@bf=\textbf\def\PY@tc##1{\textcolor[rgb]{0.00,0.50,0.00}{##1}}}
\expandafter\def\csname PY@tok@c\endcsname{\let\PY@it=\textit\def\PY@tc##1{\textcolor[rgb]{0.25,0.50,0.50}{##1}}}
\expandafter\def\csname PY@tok@mf\endcsname{\def\PY@tc##1{\textcolor[rgb]{0.40,0.40,0.40}{##1}}}
\expandafter\def\csname PY@tok@err\endcsname{\def\PY@bc##1{\setlength{\fboxsep}{0pt}\fcolorbox[rgb]{1.00,0.00,0.00}{1,1,1}{\strut ##1}}}
\expandafter\def\csname PY@tok@mb\endcsname{\def\PY@tc##1{\textcolor[rgb]{0.40,0.40,0.40}{##1}}}
\expandafter\def\csname PY@tok@ss\endcsname{\def\PY@tc##1{\textcolor[rgb]{0.10,0.09,0.49}{##1}}}
\expandafter\def\csname PY@tok@sr\endcsname{\def\PY@tc##1{\textcolor[rgb]{0.73,0.40,0.53}{##1}}}
\expandafter\def\csname PY@tok@mo\endcsname{\def\PY@tc##1{\textcolor[rgb]{0.40,0.40,0.40}{##1}}}
\expandafter\def\csname PY@tok@kd\endcsname{\let\PY@bf=\textbf\def\PY@tc##1{\textcolor[rgb]{0.00,0.50,0.00}{##1}}}
\expandafter\def\csname PY@tok@mi\endcsname{\def\PY@tc##1{\textcolor[rgb]{0.40,0.40,0.40}{##1}}}
\expandafter\def\csname PY@tok@kn\endcsname{\let\PY@bf=\textbf\def\PY@tc##1{\textcolor[rgb]{0.00,0.50,0.00}{##1}}}
\expandafter\def\csname PY@tok@cpf\endcsname{\let\PY@it=\textit\def\PY@tc##1{\textcolor[rgb]{0.25,0.50,0.50}{##1}}}
\expandafter\def\csname PY@tok@kr\endcsname{\let\PY@bf=\textbf\def\PY@tc##1{\textcolor[rgb]{0.00,0.50,0.00}{##1}}}
\expandafter\def\csname PY@tok@s\endcsname{\def\PY@tc##1{\textcolor[rgb]{0.73,0.13,0.13}{##1}}}
\expandafter\def\csname PY@tok@kp\endcsname{\def\PY@tc##1{\textcolor[rgb]{0.00,0.50,0.00}{##1}}}
\expandafter\def\csname PY@tok@w\endcsname{\def\PY@tc##1{\textcolor[rgb]{0.73,0.73,0.73}{##1}}}
\expandafter\def\csname PY@tok@kt\endcsname{\def\PY@tc##1{\textcolor[rgb]{0.69,0.00,0.25}{##1}}}
\expandafter\def\csname PY@tok@sc\endcsname{\def\PY@tc##1{\textcolor[rgb]{0.73,0.13,0.13}{##1}}}
\expandafter\def\csname PY@tok@sb\endcsname{\def\PY@tc##1{\textcolor[rgb]{0.73,0.13,0.13}{##1}}}
\expandafter\def\csname PY@tok@sa\endcsname{\def\PY@tc##1{\textcolor[rgb]{0.73,0.13,0.13}{##1}}}
\expandafter\def\csname PY@tok@k\endcsname{\let\PY@bf=\textbf\def\PY@tc##1{\textcolor[rgb]{0.00,0.50,0.00}{##1}}}
\expandafter\def\csname PY@tok@se\endcsname{\let\PY@bf=\textbf\def\PY@tc##1{\textcolor[rgb]{0.73,0.40,0.13}{##1}}}
\expandafter\def\csname PY@tok@sd\endcsname{\let\PY@it=\textit\def\PY@tc##1{\textcolor[rgb]{0.73,0.13,0.13}{##1}}}

\def\PYZbs{\char`\\}
\def\PYZus{\char`\_}
\def\PYZob{\char`\{}
\def\PYZcb{\char`\}}
\def\PYZca{\char`\^}
\def\PYZam{\char`\&}
\def\PYZlt{\char`\<}
\def\PYZgt{\char`\>}
\def\PYZsh{\char`\#}
\def\PYZpc{\char`\%}
\def\PYZdl{\char`\$}
\def\PYZhy{\char`\-}
\def\PYZsq{\char`\'}
\def\PYZdq{\char`\"}
\def\PYZti{\char`\~}
% for compatibility with earlier versions
\def\PYZat{@}
\def\PYZlb{[}
\def\PYZrb{]}
\makeatother


    % Exact colors from NB
    \definecolor{incolor}{rgb}{0.0, 0.0, 0.5}
    \definecolor{outcolor}{rgb}{0.545, 0.0, 0.0}



    
    % Prevent overflowing lines due to hard-to-break entities
    \sloppy 
    % Setup hyperref package
    \hypersetup{
      breaklinks=true,  % so long urls are correctly broken across lines
      colorlinks=true,
      urlcolor=urlcolor,
      linkcolor=linkcolor,
      citecolor=citecolor,
      }
    % Slightly bigger margins than the latex defaults
    
    \geometry{verbose,tmargin=1in,bmargin=1in,lmargin=1in,rmargin=1in}
    
    

    \begin{document}
    
    
    \maketitle
    
    

    
    \subsection{A Mental Model for Julia}\label{a-mental-model-for-julia}

Julia, while being ``a scripting language'', it is far more complex than
other scripting languages. The goal for these slides is to give a proper
mental model for approaching Julia

    \subsection{Main Reasons to Use Julia}\label{main-reasons-to-use-julia}

\begin{itemize}
\itemsep1pt\parskip0pt\parsep0pt
\item
  You want to write code which is ``slick'' and readable
\item
  Code is read more than code is written!
\item
  You want that same code to be fast
\item
  You prefer to write everything in one language
\item
  You are a developer: you want to write packages really fast which are
  really fast!
\item
  You are a package user: you heard about some really cool Julia
  packages/metapackages (JuMP, Plots, DifferentialEquations), and want
  to use Julia
\item
  Calling C, Fortran, Python, R, and MATLAB libraries, likely a
  combination of them, is necessary for your work
\item
  You want built-in and easy native parallelism
\item
  You want to build your own Domain-Specific Language (DSL), compiler,
  etc.
\end{itemize}

    \begin{figure}[htbp]
\centering
\includegraphics{https://github.com/UCIDataScienceInitiative/IntroToJulia/raw/master/assets/Julia_Interop_Test.jpg-large}
\caption{Interop}
\end{figure}

    \subsection{Syntax is Familiar}\label{syntax-is-familiar}

Higham: An Algorithmic Introduction to Numerical Simulation of
Stochastic Differential Equations

    \begin{Verbatim}[commandchars=\\\{\}]
{\color{incolor}In [{\color{incolor} }]:} \PY{o}{\PYZpc{}}\PY{n}{BPATH1} \PY{n}{Brownian} \PY{n}{path} \PY{n}{simulation}
        \PY{k}{function} \PY{p}{[}\PY{n}{t}\PY{p}{,}\PY{n}{W}\PY{p}{]}\PY{o}{=}\PY{n}{BPATH1}\PY{p}{(}\PY{n}{T}\PY{p}{,}\PY{n}{N}\PY{p}{)}
        \PY{n}{randn}\PY{p}{(}\PY{err}{\PYZsq{}}\PY{n}{state}\PY{o}{\PYZsq{}}\PY{p}{,}\PY{l+m+mi}{100}\PY{p}{)}
        \PY{n}{dt} \PY{o}{=} \PY{n}{T}\PY{o}{/}\PY{n}{N}\PY{p}{;}
        \PY{n}{dW} \PY{o}{=} \PY{n}{zeros}\PY{p}{(}\PY{l+m+mi}{10}\PY{p}{,}\PY{n}{N}\PY{p}{)}\PY{p}{;}
        \PY{n}{W} \PY{o}{=} \PY{n}{zeros}\PY{p}{(}\PY{l+m+mi}{10}\PY{p}{,}\PY{n}{N}\PY{p}{)}\PY{p}{;}
        
        \PY{n}{dW}\PY{p}{(}\PY{o}{:}\PY{p}{,}\PY{l+m+mi}{1}\PY{p}{)} \PY{o}{=} \PY{n}{sqrt}\PY{p}{(}\PY{n}{dt}\PY{p}{)}\PY{o}{*}\PY{n}{randn}\PY{p}{(}\PY{l+m+mi}{1}\PY{p}{,}\PY{l+m+mi}{10}\PY{p}{)}\PY{p}{;}
        \PY{n}{W}\PY{p}{(}\PY{o}{:}\PY{p}{,}\PY{l+m+mi}{1}\PY{p}{)} \PY{o}{=} \PY{n}{dW}\PY{p}{(}\PY{o}{:}\PY{p}{,}\PY{l+m+mi}{1}\PY{p}{)}\PY{p}{;}
        
        \PY{k}{for} \PY{n}{j} \PY{o}{=} \PY{l+m+mi}{2}\PY{o}{:}\PY{n}{N}
            \PY{n}{dW}\PY{p}{(}\PY{o}{:}\PY{p}{,}\PY{n}{j}\PY{p}{)} \PY{o}{=} \PY{n}{sqrt}\PY{p}{(}\PY{n}{dt}\PY{p}{)}\PY{o}{*}\PY{n}{randn}\PY{p}{(}\PY{l+m+mi}{1}\PY{p}{,}\PY{l+m+mi}{10}\PY{p}{)}\PY{p}{;}
            \PY{n}{W}\PY{p}{(}\PY{o}{:}\PY{p}{,}\PY{n}{j}\PY{p}{)} \PY{o}{=} \PY{n}{W}\PY{p}{(}\PY{o}{:}\PY{p}{,}\PY{n}{j}\PY{o}{\PYZhy{}}\PY{l+m+mi}{1}\PY{p}{)} \PY{o}{+} \PY{n}{dW}\PY{p}{(}\PY{o}{:}\PY{p}{,}\PY{n}{j}\PY{p}{)}\PY{p}{;}
        \PY{k}{end}
        \PY{n}{t} \PY{o}{=} \PY{p}{[}\PY{l+m+mi}{0}\PY{o}{:}\PY{n}{dt}\PY{o}{:}\PY{n}{T}\PY{o}{\PYZhy{}}\PY{n}{dt}\PY{p}{]}\PY{p}{;}
        \PY{k}{end}
        
        \PY{o}{\PYZpc{}} \PY{n}{In} \PY{n}{Another} \PY{n}{File}\PY{o}{.}\PY{o}{.}\PY{o}{.}
        
        \PY{n}{f} \PY{o}{=} \PY{err}{@}\PY{p}{(}\PY{p}{)} \PY{n}{BPATH1}\PY{p}{(}\PY{l+m+mi}{1}\PY{p}{,}\PY{l+m+mi}{10000}\PY{p}{)}\PY{p}{;}
        \PY{n}{timeit}\PY{p}{(}\PY{n}{f}\PY{p}{,}\PY{l+m+mi}{2}\PY{p}{)}
        \PY{p}{[}\PY{n}{t}\PY{p}{,}\PY{n}{W}\PY{p}{]} \PY{o}{=} \PY{n}{BPATH1}\PY{p}{(}\PY{l+m+mi}{1}\PY{p}{,}\PY{l+m+mi}{10000}\PY{p}{)}\PY{p}{;}
        \PY{n}{plot}\PY{p}{(}\PY{n}{t}\PY{p}{,}\PY{n}{W}\PY{p}{,}\PY{err}{\PYZsq{}}\PY{n}{r}\PY{o}{\PYZhy{}}\PY{err}{\PYZsq{}}\PY{p}{)}
        \PY{n}{xlabel}\PY{p}{(}\PY{l+s+sc}{\PYZsq{}t\PYZsq{}}\PY{p}{,}\PY{err}{\PYZsq{}}\PY{n}{FontSize}\PY{o}{\PYZsq{}}\PY{p}{,}\PY{l+m+mi}{16}\PY{p}{)}
        \PY{n}{ylabel}\PY{p}{(}\PY{err}{\PYZsq{}}\PY{n}{W}\PY{p}{(}\PY{n}{t}\PY{p}{)}\PY{l+s+sc}{\PYZsq{},\PYZsq{}}\PY{n}{FontSize}\PY{o}{\PYZsq{}}\PY{p}{,}\PY{l+m+mi}{16}\PY{p}{,}\PY{err}{\PYZsq{}}\PY{n}{Rotation}\PY{o}{\PYZsq{}}\PY{p}{,}\PY{l+m+mi}{0}\PY{p}{)}
\end{Verbatim}

    \begin{Verbatim}[commandchars=\\\{\}]
{\color{incolor}In [{\color{incolor}1}]:} \PY{k}{function} \PY{n}{bpath}\PY{p}{(}\PY{n}{T}\PY{p}{,}\PY{n}{N}\PY{p}{)}
            \PY{n}{srand}\PY{p}{(}\PY{l+m+mi}{100}\PY{p}{)}
            \PY{n}{dt} \PY{o}{=} \PY{n}{T}\PY{o}{/}\PY{n}{N}
            \PY{n}{dW} \PY{o}{=} \PY{n}{zeros}\PY{p}{(}\PY{l+m+mi}{10}\PY{p}{,}\PY{n}{N}\PY{p}{)}
            \PY{n}{W} \PY{o}{=} \PY{n}{zeros}\PY{p}{(}\PY{l+m+mi}{10}\PY{p}{,}\PY{n}{N}\PY{p}{)}
        
            \PY{n}{dW}\PY{p}{[}\PY{o}{:}\PY{p}{,}\PY{l+m+mi}{1}\PY{p}{]} \PY{o}{=} \PY{n}{sqrt}\PY{p}{(}\PY{n}{dt}\PY{p}{)}\PY{o}{*}\PY{n}{randn}\PY{p}{(}\PY{l+m+mi}{10}\PY{p}{)}
            \PY{n}{W}\PY{p}{[}\PY{o}{:}\PY{p}{,}\PY{l+m+mi}{1}\PY{p}{]} \PY{o}{=} \PY{n}{dW}\PY{p}{[}\PY{o}{:}\PY{p}{,}\PY{l+m+mi}{1}\PY{p}{]}
        
            \PY{k}{for} \PY{n}{j} \PY{o}{=} \PY{l+m+mi}{2}\PY{o}{:}\PY{n}{N}
                \PY{n}{dW}\PY{p}{[}\PY{o}{:}\PY{p}{,}\PY{n}{j}\PY{p}{]} \PY{o}{.=} \PY{n}{sqrt}\PY{p}{(}\PY{n}{dt}\PY{p}{)}\PY{o}{*}\PY{n}{randn}\PY{p}{(}\PY{l+m+mi}{10}\PY{p}{)}
                \PY{n}{W}\PY{p}{[}\PY{o}{:}\PY{p}{,}\PY{n}{j}\PY{p}{]} \PY{o}{.=} \PY{n}{W}\PY{p}{[}\PY{o}{:}\PY{p}{,}\PY{n}{j}\PY{o}{\PYZhy{}}\PY{l+m+mi}{1}\PY{p}{]} \PY{o}{.+} \PY{n}{dW}\PY{p}{[}\PY{o}{:}\PY{p}{,}\PY{n}{j}\PY{p}{]}
            \PY{k}{end}
            \PY{p}{[}\PY{l+m+mi}{0}\PY{o}{:}\PY{n}{dt}\PY{o}{:}\PY{n}{T}\PY{o}{\PYZhy{}}\PY{n}{dt}\PY{p}{]}\PY{p}{,}\PY{n}{W}
        \PY{k}{end} \PY{c}{\PYZsh{} Translation took \PYZlt{} 1 minute}
        \PY{n+nd}{@time} \PY{n}{t}\PY{p}{,}\PY{n}{W} \PY{o}{=} \PY{n}{bpath}\PY{p}{(}\PY{l+m+mi}{1}\PY{p}{,}\PY{l+m+mi}{100000}\PY{p}{)} \PY{c}{\PYZsh{} Timing is slower in Jupyter!}
        \PY{n+nd}{@time} \PY{n}{t}\PY{p}{,}\PY{n}{W} \PY{o}{=} \PY{n}{bpath}\PY{p}{(}\PY{l+m+mi}{1}\PY{p}{,}\PY{l+m+mi}{100000}\PY{p}{)} \PY{c}{\PYZsh{} Timing is slower in Jupyter!}
        \PY{c}{\PYZsh{} 10x speedup over MATLAB}
        \PY{k}{using} \PY{n}{Plots}
        \PY{n}{plot}\PY{p}{(}\PY{n}{t}\PY{p}{,}\PY{n}{W}\PY{o}{\PYZsq{}}\PY{p}{,}\PY{n}{color}\PY{o}{=}\PY{o}{:}\PY{n}{red}\PY{p}{,}\PY{n}{xlabel}\PY{o}{=}\PY{l+s}{\PYZdq{}}\PY{l+s}{t}\PY{l+s}{\PYZdq{}}\PY{p}{,}\PY{n}{ylabel}\PY{o}{=}\PY{l+s}{\PYZdq{}}\PY{l+s}{W}\PY{l+s}{(}\PY{l+s}{t}\PY{l+s}{)}\PY{l+s}{\PYZdq{}}\PY{p}{)}
\end{Verbatim}

    \begin{Verbatim}[commandchars=\\\{\}]
  0.312249 seconds (1.58 M allocations: 123.550 MB, 5.14\% gc time)
  0.051225 seconds (1.30 M allocations: 111.374 MB, 16.63\% gc time)

    \end{Verbatim}

    \subsection{\href{http://cheatsheets.quantecon.org/}{MATLAB, Python,
Julia Syntax Comparison}}\label{matlab-python-julia-syntax-comparison}

    \subsection{Main Reasons to Not Use
Julia}\label{main-reasons-to-not-use-julia}

\begin{itemize}
\itemsep1pt\parskip0pt\parsep0pt
\item
  You need low latency (game programming)
\item
  You need stability. i.e.~you are a company and need a large code-base
  to run without modifications next year
\item
  It is more complex than other scripting languages: there is a lot you
  can know
\end{itemize}

    \subsection{The Julia Community: Who is a
user?}\label{the-julia-community-who-is-a-user}

\begin{itemize}
\itemsep1pt\parskip0pt\parsep0pt
\item
  Julia, being high performance and equipped with heavy ``CS'' features,
  all while a scripting language, has attracted a diverse audience.
\item
  A large group of Julia users are ex-MATLAB users interested in using
  Julia for faster numerical linear algebra applications.
\item
  Another significant group are the machine learning and statistics
  users from R/Python looking to solve the ``two-language'' problem.
\item
  Another group (a lot of Base contributors) are C/Fortran developers
  who are looking to increase productivity without sacrificing speed.
\item
  Another group are ``general-purpose'' users: using Julia to develop
  faster web frameworks, compilers, and anyting else you can think of!
\item
  Another group is from functional programming languages (Haskell) and
  Lisps (Clojure, Femtolisp) interested in Julia's metaprogramming and
  parallelism.
\end{itemize}

Julia combines the interests, features, and libraries of all of these
groups. To which group(s) do you belong?

    We want a language that's open source, with a liberal license. We want
the speed of C with the dynamism of Ruby. We want a language that's
homoiconic, with true macros like Lisp, but with obvious, familiar
mathematical notation like Matlab. We want something as usable for
general programming as Python, as easy for statistics as R, as natural
for string processing as Perl, as powerful for linear algebra as Matlab,
as good at gluing programs together as the shell. Something that is dirt
simple to learn, yet keeps the most serious hackers happy. We want it
interactive and we want it compiled.

(Did we mention it should be as fast as C?)

While we're being demanding, we want something that provides the
distributed power of Hadoop --- without the kilobytes of boilerplate
Java and XML; without being forced to sift through gigabytes of log
files on hundreds of machines to find our bugs. We want the power
without the layers of impenetrable complexity. We want to write simple
scalar loops that compile down to tight machine code using just the
registers on a single CPU. We want to write A*B and launch a thousand
computations on a thousand machines, calculating a vast matrix product
together.

    \subsection{Who Julia is Developed
For}\label{who-julia-is-developed-for}

\begin{itemize}
\itemsep1pt\parskip0pt\parsep0pt
\item
  The Julia project has one of the largest numbers of contributors of
  any open-source language.
\item
  There have been 500+ contributors, with 100+ having made more than 10
  commits.
\item
  Julia is created by people spanning these groups.
\item
  Julia is made to have innovative design, but not syntax.
\end{itemize}

Julia's syntax is pulled from places considered ``best-in-class'':

\begin{itemize}
\itemsep1pt\parskip0pt\parsep0pt
\item
  The statistics looks like R.
\item
  The linear algebra looks like MATLAB.
\item
  The general-purpose parts look like Python.
\item
  The macros work like Lisp (they can look like it using
  LispSyntax.jl!).
\item
  The fast devectorized code acts like C/Fortran.
\item
  Built-in support for piping and first-class functions lets some people
  program Julia functionally.
\end{itemize}

No matter who you are, some of the terminology will be familiar, while
other parts will be pulled from a domain you may have never heard of.

    \subsection{A Mental Model for Python/R/MATLAB: Talking to a
Politician}\label{a-mental-model-for-pythonrmatlab-talking-to-a-politician}

\begin{itemize}
\itemsep1pt\parskip0pt\parsep0pt
\item
  These scripting languages were developed to ``be easy''.
\item
  You tell them something, and they try to give you want you want.
\item
  There may be some things hidden behind the scenes to make everything
  ``work better''.
\item
  They may not give you the fastest reply.
\end{itemize}

    \subsection{A Mental Model for C/Fortran: Talking to a
Philosopher}\label{a-mental-model-for-cfortran-talking-to-a-philosopher}

\begin{itemize}
\itemsep1pt\parskip0pt\parsep0pt
\item
  You say something, and they want something more specific.
\item
  You spend hours digging deep into the specifics of something.
\item
  After finally getting it right, you know how to quickly get a specific
  answer from them.
\item
  Everytime you want to talk about something new, you have to start all
  the way at the basics again.
\end{itemize}

    \subsection{A Mental Model for Julia: Talking to a
Scientist}\label{a-mental-model-for-julia-talking-to-a-scientist}

\begin{itemize}
\itemsep1pt\parskip0pt\parsep0pt
\item
  When you're talking, everything looks general. However, you really
  mean very specific details determined by context.
\item
  You can quickly dig deep into a subject, assuming many rules,
  theories, and terminology.
\item
  Nothing is hidden: if you ever want to hear about every little detail,
  you can ask.
\item
  They will get mad (and throw errors at you) if you begin to be loose
  with the specific details.
\end{itemize}

Conclusion: While Julia looks at the surface like R/Python/MATLAB,
what's actually happening under the hood is very different. It is this
design difference which is essential to getting the full performance out
of Julia while not sacrificing readability. The goal of the workshop is
to show you how this works, and how to make it work to your advantage.

    \subsection{The Rabbit Hole and
Misconceptions}\label{the-rabbit-hole-and-misconceptions}

Since Julia has so many different influences meshed together, there is a
``rabbit hole'' of features to explore, designs to investigate, and
performance tricks to exploit. We will be peaking into the rabbit hole.
Here's a good snippet of a podcast that addresses some misconceptions.

http://www.rce-cast.com/

Karpinski: The syntax is superficially similar to MATLAB, so you can
often translate MATLAB code to Julia just by changing a few parenthesis
to square brackets for indexing into arrays and not really changing too
much else. But the symantics are closest probably to Python: it's
{[}a{]} very straightforward dynamic language to write use. But then
there's sort of this rabbit hole of advanced features that you can go
down that you don't need to know about right away to write useful
programs, but which can help you as you find yourself doing harder and
harder things.

Edelman: What happens when you start to go down this rabbit hole is you
become programmer, something for when you used these other languages you
never knew you were missing, and never knew you wanted to be. But then
when you do it, you wonder how you lived without it.

Listen to 5:50 - 17:00


    % Add a bibliography block to the postdoc
    
    
    
    \end{document}
