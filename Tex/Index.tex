
% Default to the notebook output style

    


% Inherit from the specified cell style.




    
\documentclass[11pt]{article}

    
    
    \usepackage[T1]{fontenc}
    % Nicer default font (+ math font) than Computer Modern for most use cases
    \usepackage{mathpazo}

    % Basic figure setup, for now with no caption control since it's done
    % automatically by Pandoc (which extracts ![](path) syntax from Markdown).
    \usepackage{graphicx}
    % We will generate all images so they have a width \maxwidth. This means
    % that they will get their normal width if they fit onto the page, but
    % are scaled down if they would overflow the margins.
    \makeatletter
    \def\maxwidth{\ifdim\Gin@nat@width>\linewidth\linewidth
    \else\Gin@nat@width\fi}
    \makeatother
    \let\Oldincludegraphics\includegraphics
    % Set max figure width to be 80% of text width, for now hardcoded.
    \renewcommand{\includegraphics}[1]{\Oldincludegraphics[width=.8\maxwidth]{#1}}
    % Ensure that by default, figures have no caption (until we provide a
    % proper Figure object with a Caption API and a way to capture that
    % in the conversion process - todo).
    \usepackage{caption}
    \DeclareCaptionLabelFormat{nolabel}{}
    \captionsetup{labelformat=nolabel}

    \usepackage{adjustbox} % Used to constrain images to a maximum size 
    \usepackage{xcolor} % Allow colors to be defined
    \usepackage{enumerate} % Needed for markdown enumerations to work
    \usepackage{geometry} % Used to adjust the document margins
    \usepackage{amsmath} % Equations
    \usepackage{amssymb} % Equations
    \usepackage{textcomp} % defines textquotesingle
    % Hack from http://tex.stackexchange.com/a/47451/13684:
    \AtBeginDocument{%
        \def\PYZsq{\textquotesingle}% Upright quotes in Pygmentized code
    }
    \usepackage{upquote} % Upright quotes for verbatim code
    \usepackage{eurosym} % defines \euro
    \usepackage[mathletters]{ucs} % Extended unicode (utf-8) support
    \usepackage[utf8x]{inputenc} % Allow utf-8 characters in the tex document
    \usepackage{fancyvrb} % verbatim replacement that allows latex
    \usepackage{grffile} % extends the file name processing of package graphics 
                         % to support a larger range 
    % The hyperref package gives us a pdf with properly built
    % internal navigation ('pdf bookmarks' for the table of contents,
    % internal cross-reference links, web links for URLs, etc.)
    \usepackage{hyperref}
    \usepackage{longtable} % longtable support required by pandoc >1.10
    \usepackage{booktabs}  % table support for pandoc > 1.12.2
    \usepackage[inline]{enumitem} % IRkernel/repr support (it uses the enumerate* environment)
    \usepackage[normalem]{ulem} % ulem is needed to support strikethroughs (\sout)
                                % normalem makes italics be italics, not underlines
    

    
    
    % Colors for the hyperref package
    \definecolor{urlcolor}{rgb}{0,.145,.698}
    \definecolor{linkcolor}{rgb}{.71,0.21,0.01}
    \definecolor{citecolor}{rgb}{.12,.54,.11}

    % ANSI colors
    \definecolor{ansi-black}{HTML}{3E424D}
    \definecolor{ansi-black-intense}{HTML}{282C36}
    \definecolor{ansi-red}{HTML}{E75C58}
    \definecolor{ansi-red-intense}{HTML}{B22B31}
    \definecolor{ansi-green}{HTML}{00A250}
    \definecolor{ansi-green-intense}{HTML}{007427}
    \definecolor{ansi-yellow}{HTML}{DDB62B}
    \definecolor{ansi-yellow-intense}{HTML}{B27D12}
    \definecolor{ansi-blue}{HTML}{208FFB}
    \definecolor{ansi-blue-intense}{HTML}{0065CA}
    \definecolor{ansi-magenta}{HTML}{D160C4}
    \definecolor{ansi-magenta-intense}{HTML}{A03196}
    \definecolor{ansi-cyan}{HTML}{60C6C8}
    \definecolor{ansi-cyan-intense}{HTML}{258F8F}
    \definecolor{ansi-white}{HTML}{C5C1B4}
    \definecolor{ansi-white-intense}{HTML}{A1A6B2}

    % commands and environments needed by pandoc snippets
    % extracted from the output of `pandoc -s`
    \providecommand{\tightlist}{%
      \setlength{\itemsep}{0pt}\setlength{\parskip}{0pt}}
    \DefineVerbatimEnvironment{Highlighting}{Verbatim}{commandchars=\\\{\}}
    % Add ',fontsize=\small' for more characters per line
    \newenvironment{Shaded}{}{}
    \newcommand{\KeywordTok}[1]{\textcolor[rgb]{0.00,0.44,0.13}{\textbf{{#1}}}}
    \newcommand{\DataTypeTok}[1]{\textcolor[rgb]{0.56,0.13,0.00}{{#1}}}
    \newcommand{\DecValTok}[1]{\textcolor[rgb]{0.25,0.63,0.44}{{#1}}}
    \newcommand{\BaseNTok}[1]{\textcolor[rgb]{0.25,0.63,0.44}{{#1}}}
    \newcommand{\FloatTok}[1]{\textcolor[rgb]{0.25,0.63,0.44}{{#1}}}
    \newcommand{\CharTok}[1]{\textcolor[rgb]{0.25,0.44,0.63}{{#1}}}
    \newcommand{\StringTok}[1]{\textcolor[rgb]{0.25,0.44,0.63}{{#1}}}
    \newcommand{\CommentTok}[1]{\textcolor[rgb]{0.38,0.63,0.69}{\textit{{#1}}}}
    \newcommand{\OtherTok}[1]{\textcolor[rgb]{0.00,0.44,0.13}{{#1}}}
    \newcommand{\AlertTok}[1]{\textcolor[rgb]{1.00,0.00,0.00}{\textbf{{#1}}}}
    \newcommand{\FunctionTok}[1]{\textcolor[rgb]{0.02,0.16,0.49}{{#1}}}
    \newcommand{\RegionMarkerTok}[1]{{#1}}
    \newcommand{\ErrorTok}[1]{\textcolor[rgb]{1.00,0.00,0.00}{\textbf{{#1}}}}
    \newcommand{\NormalTok}[1]{{#1}}
    
    % Additional commands for more recent versions of Pandoc
    \newcommand{\ConstantTok}[1]{\textcolor[rgb]{0.53,0.00,0.00}{{#1}}}
    \newcommand{\SpecialCharTok}[1]{\textcolor[rgb]{0.25,0.44,0.63}{{#1}}}
    \newcommand{\VerbatimStringTok}[1]{\textcolor[rgb]{0.25,0.44,0.63}{{#1}}}
    \newcommand{\SpecialStringTok}[1]{\textcolor[rgb]{0.73,0.40,0.53}{{#1}}}
    \newcommand{\ImportTok}[1]{{#1}}
    \newcommand{\DocumentationTok}[1]{\textcolor[rgb]{0.73,0.13,0.13}{\textit{{#1}}}}
    \newcommand{\AnnotationTok}[1]{\textcolor[rgb]{0.38,0.63,0.69}{\textbf{\textit{{#1}}}}}
    \newcommand{\CommentVarTok}[1]{\textcolor[rgb]{0.38,0.63,0.69}{\textbf{\textit{{#1}}}}}
    \newcommand{\VariableTok}[1]{\textcolor[rgb]{0.10,0.09,0.49}{{#1}}}
    \newcommand{\ControlFlowTok}[1]{\textcolor[rgb]{0.00,0.44,0.13}{\textbf{{#1}}}}
    \newcommand{\OperatorTok}[1]{\textcolor[rgb]{0.40,0.40,0.40}{{#1}}}
    \newcommand{\BuiltInTok}[1]{{#1}}
    \newcommand{\ExtensionTok}[1]{{#1}}
    \newcommand{\PreprocessorTok}[1]{\textcolor[rgb]{0.74,0.48,0.00}{{#1}}}
    \newcommand{\AttributeTok}[1]{\textcolor[rgb]{0.49,0.56,0.16}{{#1}}}
    \newcommand{\InformationTok}[1]{\textcolor[rgb]{0.38,0.63,0.69}{\textbf{\textit{{#1}}}}}
    \newcommand{\WarningTok}[1]{\textcolor[rgb]{0.38,0.63,0.69}{\textbf{\textit{{#1}}}}}
    
    
    % Define a nice break command that doesn't care if a line doesn't already
    % exist.
    \def\br{\hspace*{\fill} \\* }
    % Math Jax compatability definitions
    \def\gt{>}
    \def\lt{<}
    % Document parameters
    \title{Index}
    
    
    

    % Pygments definitions
    
\makeatletter
\def\PY@reset{\let\PY@it=\relax \let\PY@bf=\relax%
    \let\PY@ul=\relax \let\PY@tc=\relax%
    \let\PY@bc=\relax \let\PY@ff=\relax}
\def\PY@tok#1{\csname PY@tok@#1\endcsname}
\def\PY@toks#1+{\ifx\relax#1\empty\else%
    \PY@tok{#1}\expandafter\PY@toks\fi}
\def\PY@do#1{\PY@bc{\PY@tc{\PY@ul{%
    \PY@it{\PY@bf{\PY@ff{#1}}}}}}}
\def\PY#1#2{\PY@reset\PY@toks#1+\relax+\PY@do{#2}}

\expandafter\def\csname PY@tok@w\endcsname{\def\PY@tc##1{\textcolor[rgb]{0.73,0.73,0.73}{##1}}}
\expandafter\def\csname PY@tok@c\endcsname{\let\PY@it=\textit\def\PY@tc##1{\textcolor[rgb]{0.25,0.50,0.50}{##1}}}
\expandafter\def\csname PY@tok@cp\endcsname{\def\PY@tc##1{\textcolor[rgb]{0.74,0.48,0.00}{##1}}}
\expandafter\def\csname PY@tok@k\endcsname{\let\PY@bf=\textbf\def\PY@tc##1{\textcolor[rgb]{0.00,0.50,0.00}{##1}}}
\expandafter\def\csname PY@tok@kp\endcsname{\def\PY@tc##1{\textcolor[rgb]{0.00,0.50,0.00}{##1}}}
\expandafter\def\csname PY@tok@kt\endcsname{\def\PY@tc##1{\textcolor[rgb]{0.69,0.00,0.25}{##1}}}
\expandafter\def\csname PY@tok@o\endcsname{\def\PY@tc##1{\textcolor[rgb]{0.40,0.40,0.40}{##1}}}
\expandafter\def\csname PY@tok@ow\endcsname{\let\PY@bf=\textbf\def\PY@tc##1{\textcolor[rgb]{0.67,0.13,1.00}{##1}}}
\expandafter\def\csname PY@tok@nb\endcsname{\def\PY@tc##1{\textcolor[rgb]{0.00,0.50,0.00}{##1}}}
\expandafter\def\csname PY@tok@nf\endcsname{\def\PY@tc##1{\textcolor[rgb]{0.00,0.00,1.00}{##1}}}
\expandafter\def\csname PY@tok@nc\endcsname{\let\PY@bf=\textbf\def\PY@tc##1{\textcolor[rgb]{0.00,0.00,1.00}{##1}}}
\expandafter\def\csname PY@tok@nn\endcsname{\let\PY@bf=\textbf\def\PY@tc##1{\textcolor[rgb]{0.00,0.00,1.00}{##1}}}
\expandafter\def\csname PY@tok@ne\endcsname{\let\PY@bf=\textbf\def\PY@tc##1{\textcolor[rgb]{0.82,0.25,0.23}{##1}}}
\expandafter\def\csname PY@tok@nv\endcsname{\def\PY@tc##1{\textcolor[rgb]{0.10,0.09,0.49}{##1}}}
\expandafter\def\csname PY@tok@no\endcsname{\def\PY@tc##1{\textcolor[rgb]{0.53,0.00,0.00}{##1}}}
\expandafter\def\csname PY@tok@nl\endcsname{\def\PY@tc##1{\textcolor[rgb]{0.63,0.63,0.00}{##1}}}
\expandafter\def\csname PY@tok@ni\endcsname{\let\PY@bf=\textbf\def\PY@tc##1{\textcolor[rgb]{0.60,0.60,0.60}{##1}}}
\expandafter\def\csname PY@tok@na\endcsname{\def\PY@tc##1{\textcolor[rgb]{0.49,0.56,0.16}{##1}}}
\expandafter\def\csname PY@tok@nt\endcsname{\let\PY@bf=\textbf\def\PY@tc##1{\textcolor[rgb]{0.00,0.50,0.00}{##1}}}
\expandafter\def\csname PY@tok@nd\endcsname{\def\PY@tc##1{\textcolor[rgb]{0.67,0.13,1.00}{##1}}}
\expandafter\def\csname PY@tok@s\endcsname{\def\PY@tc##1{\textcolor[rgb]{0.73,0.13,0.13}{##1}}}
\expandafter\def\csname PY@tok@sd\endcsname{\let\PY@it=\textit\def\PY@tc##1{\textcolor[rgb]{0.73,0.13,0.13}{##1}}}
\expandafter\def\csname PY@tok@si\endcsname{\let\PY@bf=\textbf\def\PY@tc##1{\textcolor[rgb]{0.73,0.40,0.53}{##1}}}
\expandafter\def\csname PY@tok@se\endcsname{\let\PY@bf=\textbf\def\PY@tc##1{\textcolor[rgb]{0.73,0.40,0.13}{##1}}}
\expandafter\def\csname PY@tok@sr\endcsname{\def\PY@tc##1{\textcolor[rgb]{0.73,0.40,0.53}{##1}}}
\expandafter\def\csname PY@tok@ss\endcsname{\def\PY@tc##1{\textcolor[rgb]{0.10,0.09,0.49}{##1}}}
\expandafter\def\csname PY@tok@sx\endcsname{\def\PY@tc##1{\textcolor[rgb]{0.00,0.50,0.00}{##1}}}
\expandafter\def\csname PY@tok@m\endcsname{\def\PY@tc##1{\textcolor[rgb]{0.40,0.40,0.40}{##1}}}
\expandafter\def\csname PY@tok@gh\endcsname{\let\PY@bf=\textbf\def\PY@tc##1{\textcolor[rgb]{0.00,0.00,0.50}{##1}}}
\expandafter\def\csname PY@tok@gu\endcsname{\let\PY@bf=\textbf\def\PY@tc##1{\textcolor[rgb]{0.50,0.00,0.50}{##1}}}
\expandafter\def\csname PY@tok@gd\endcsname{\def\PY@tc##1{\textcolor[rgb]{0.63,0.00,0.00}{##1}}}
\expandafter\def\csname PY@tok@gi\endcsname{\def\PY@tc##1{\textcolor[rgb]{0.00,0.63,0.00}{##1}}}
\expandafter\def\csname PY@tok@gr\endcsname{\def\PY@tc##1{\textcolor[rgb]{1.00,0.00,0.00}{##1}}}
\expandafter\def\csname PY@tok@ge\endcsname{\let\PY@it=\textit}
\expandafter\def\csname PY@tok@gs\endcsname{\let\PY@bf=\textbf}
\expandafter\def\csname PY@tok@gp\endcsname{\let\PY@bf=\textbf\def\PY@tc##1{\textcolor[rgb]{0.00,0.00,0.50}{##1}}}
\expandafter\def\csname PY@tok@go\endcsname{\def\PY@tc##1{\textcolor[rgb]{0.53,0.53,0.53}{##1}}}
\expandafter\def\csname PY@tok@gt\endcsname{\def\PY@tc##1{\textcolor[rgb]{0.00,0.27,0.87}{##1}}}
\expandafter\def\csname PY@tok@err\endcsname{\def\PY@bc##1{\setlength{\fboxsep}{0pt}\fcolorbox[rgb]{1.00,0.00,0.00}{1,1,1}{\strut ##1}}}
\expandafter\def\csname PY@tok@kc\endcsname{\let\PY@bf=\textbf\def\PY@tc##1{\textcolor[rgb]{0.00,0.50,0.00}{##1}}}
\expandafter\def\csname PY@tok@kd\endcsname{\let\PY@bf=\textbf\def\PY@tc##1{\textcolor[rgb]{0.00,0.50,0.00}{##1}}}
\expandafter\def\csname PY@tok@kn\endcsname{\let\PY@bf=\textbf\def\PY@tc##1{\textcolor[rgb]{0.00,0.50,0.00}{##1}}}
\expandafter\def\csname PY@tok@kr\endcsname{\let\PY@bf=\textbf\def\PY@tc##1{\textcolor[rgb]{0.00,0.50,0.00}{##1}}}
\expandafter\def\csname PY@tok@bp\endcsname{\def\PY@tc##1{\textcolor[rgb]{0.00,0.50,0.00}{##1}}}
\expandafter\def\csname PY@tok@fm\endcsname{\def\PY@tc##1{\textcolor[rgb]{0.00,0.00,1.00}{##1}}}
\expandafter\def\csname PY@tok@vc\endcsname{\def\PY@tc##1{\textcolor[rgb]{0.10,0.09,0.49}{##1}}}
\expandafter\def\csname PY@tok@vg\endcsname{\def\PY@tc##1{\textcolor[rgb]{0.10,0.09,0.49}{##1}}}
\expandafter\def\csname PY@tok@vi\endcsname{\def\PY@tc##1{\textcolor[rgb]{0.10,0.09,0.49}{##1}}}
\expandafter\def\csname PY@tok@vm\endcsname{\def\PY@tc##1{\textcolor[rgb]{0.10,0.09,0.49}{##1}}}
\expandafter\def\csname PY@tok@sa\endcsname{\def\PY@tc##1{\textcolor[rgb]{0.73,0.13,0.13}{##1}}}
\expandafter\def\csname PY@tok@sb\endcsname{\def\PY@tc##1{\textcolor[rgb]{0.73,0.13,0.13}{##1}}}
\expandafter\def\csname PY@tok@sc\endcsname{\def\PY@tc##1{\textcolor[rgb]{0.73,0.13,0.13}{##1}}}
\expandafter\def\csname PY@tok@dl\endcsname{\def\PY@tc##1{\textcolor[rgb]{0.73,0.13,0.13}{##1}}}
\expandafter\def\csname PY@tok@s2\endcsname{\def\PY@tc##1{\textcolor[rgb]{0.73,0.13,0.13}{##1}}}
\expandafter\def\csname PY@tok@sh\endcsname{\def\PY@tc##1{\textcolor[rgb]{0.73,0.13,0.13}{##1}}}
\expandafter\def\csname PY@tok@s1\endcsname{\def\PY@tc##1{\textcolor[rgb]{0.73,0.13,0.13}{##1}}}
\expandafter\def\csname PY@tok@mb\endcsname{\def\PY@tc##1{\textcolor[rgb]{0.40,0.40,0.40}{##1}}}
\expandafter\def\csname PY@tok@mf\endcsname{\def\PY@tc##1{\textcolor[rgb]{0.40,0.40,0.40}{##1}}}
\expandafter\def\csname PY@tok@mh\endcsname{\def\PY@tc##1{\textcolor[rgb]{0.40,0.40,0.40}{##1}}}
\expandafter\def\csname PY@tok@mi\endcsname{\def\PY@tc##1{\textcolor[rgb]{0.40,0.40,0.40}{##1}}}
\expandafter\def\csname PY@tok@il\endcsname{\def\PY@tc##1{\textcolor[rgb]{0.40,0.40,0.40}{##1}}}
\expandafter\def\csname PY@tok@mo\endcsname{\def\PY@tc##1{\textcolor[rgb]{0.40,0.40,0.40}{##1}}}
\expandafter\def\csname PY@tok@ch\endcsname{\let\PY@it=\textit\def\PY@tc##1{\textcolor[rgb]{0.25,0.50,0.50}{##1}}}
\expandafter\def\csname PY@tok@cm\endcsname{\let\PY@it=\textit\def\PY@tc##1{\textcolor[rgb]{0.25,0.50,0.50}{##1}}}
\expandafter\def\csname PY@tok@cpf\endcsname{\let\PY@it=\textit\def\PY@tc##1{\textcolor[rgb]{0.25,0.50,0.50}{##1}}}
\expandafter\def\csname PY@tok@c1\endcsname{\let\PY@it=\textit\def\PY@tc##1{\textcolor[rgb]{0.25,0.50,0.50}{##1}}}
\expandafter\def\csname PY@tok@cs\endcsname{\let\PY@it=\textit\def\PY@tc##1{\textcolor[rgb]{0.25,0.50,0.50}{##1}}}

\def\PYZbs{\char`\\}
\def\PYZus{\char`\_}
\def\PYZob{\char`\{}
\def\PYZcb{\char`\}}
\def\PYZca{\char`\^}
\def\PYZam{\char`\&}
\def\PYZlt{\char`\<}
\def\PYZgt{\char`\>}
\def\PYZsh{\char`\#}
\def\PYZpc{\char`\%}
\def\PYZdl{\char`\$}
\def\PYZhy{\char`\-}
\def\PYZsq{\char`\'}
\def\PYZdq{\char`\"}
\def\PYZti{\char`\~}
% for compatibility with earlier versions
\def\PYZat{@}
\def\PYZlb{[}
\def\PYZrb{]}
\makeatother


    % Exact colors from NB
    \definecolor{incolor}{rgb}{0.0, 0.0, 0.5}
    \definecolor{outcolor}{rgb}{0.545, 0.0, 0.0}



    
    % Prevent overflowing lines due to hard-to-break entities
    \sloppy 
    % Setup hyperref package
    \hypersetup{
      breaklinks=true,  % so long urls are correctly broken across lines
      colorlinks=true,
      urlcolor=urlcolor,
      linkcolor=linkcolor,
      citecolor=citecolor,
      }
    % Slightly bigger margins than the latex defaults
    
    \geometry{verbose,tmargin=1in,bmargin=1in,lmargin=1in,rmargin=1in}
    
    

    \begin{document}
    
    
    \maketitle
    
    

    
    \section{A Deep Introduction to Julia for Data Science and Scientific
Computing}\label{a-deep-introduction-to-julia-for-data-science-and-scientific-computing}

    \subsection{Introduction}\label{introduction}

This workshop is put together by Chris Rackauckas as part of the UC
Irvine Data Science Initiative. This workshop is made to teach people
who are experienced with other scripting languages the relatively new
language Julia. Unlike the other Data Science Initiative workshops, this
workshop assumes prior knowledge of some form of programming in a
language such as Python, R, or MATLAB.

We will start by introducing the basic syntax of the language, and
quickly move into the details of how Julia is different from other
scripting languages and how to exploit Julia's type system + multiple
dispatch to be able to achieve C/Fortran-like performance while
maintaining the concise syntax of a scripting language. Large parts of
the package ecosystem will be explored and some information on
implementation details will be highlighted in order for the participants
to learn how to design Julia projects.

The ideal participant is anyone who is interested in Julia. There are
many groups of people interested in using Julia. One large fraction come
with a strong software development background and C/Fortran knowledge,
and are looking to learn Julia as a tool to create packages with
enhanced productivity while not losing performance. On the other side,
there are users who are interested in the growing package ecosystem of
Julia and would like to add Julia to their knowledge-base. And then
there's everything in between. One major goal of this workshop is to use
Julia's language and syntax to bridge the gap between "package users"
and "package developers" in the way that Julia has done.

\paragraph{This is very problem-focused. The method is not passive: the
goal is to get you using
Julia!}\label{this-is-very-problem-focused.-the-method-is-not-passive-the-goal-is-to-get-you-using-julia}

    \subsection{Introduction to the
Author}\label{introduction-to-the-author}

Chris is a PhD student in Mathematics from the Mathematical,
Computational, and Systems Biology (MCSB) Gateway program and is an
active part of the Julia community. He runs the blog,
http://www.stochasticlifestyle.com/, where he write tutorials on using
the Julia languages covering many topics, such as high-performance and
GPGPU computing, package development, and Julia tips. He is part of the
JuliaMath and JuliaDiffEq communities for curating the Julia libraries
for mathematics and differential equations respectively. Chris is the
developer of many Julia packages, the most prominent being
\href{https://github.com/JuliaDiffEq/DifferentialEquations.jl}{DifferentialEquations.jl},
a Julia library which has become a standard solver for many forms of
differential equations.

    \subsection{Notes Before We Get
Started}\label{notes-before-we-get-started}

\begin{itemize}
\tightlist
\item
  Please install some version of IJulia/Jupyter to follow along. I also
  recommend working through longer problems using the Juno IDE.
\item
  The start of the course will be on developing general performant Julia
  code. After lunch it will be about the package ecosystem. This
  understanding of Julia will be useful even for "Julia users" (i.e. non
  package developers) to use packages effectively, but don't worry we
  will get to packages.
\item
  Please do the problems/projects at your own pace. Not everyone is
  expected to complete all of the material in the allotted time.
  \textbf{Some of the problems are supposed to be hard!} Instead, this
  is supposed to be a resource to introduce you to a large amount of
  Julia, and the material may take awhile to fully be digested. That's
  okay!
\item
  During the basic introduction, there will be information that is not
  included in these notebooks. That is intentional. One major hurdle for
  learning a language is learning how to find out more about the
  language. Please use the manual, chatrooms, StackExchange, etc. If
  these aren't working for you, ask for help.
\end{itemize}

A good primer for the workshop:
https://www.youtube.com/watch?v=JNvMs0j3a4E

    \subsection{Installation and Setup}\label{installation-and-setup}

To get started, see the
\href{http://ucidatascienceinitiative.github.io/IntroToJulia/Html/ToolingDocumentationCommunity}{Tooling,
Documentation, and Community notebook}.

    \subsection{Rendered Jupyter
Notebooks}\label{rendered-jupyter-notebooks}

\subsubsection{Introduction}\label{introduction}

\begin{itemize}
\tightlist
\item
  \href{http://ucidatascienceinitiative.github.io/IntroToJulia/Html/ToolingDocumentationCommunity}{Tooling,
  Documentation, and Community}
\item
  \href{http://docs.junolab.org/latest/}{Juno, the Julia IDE (Debugging,
  Progress Bars, etc.): JunoLab}
\item
  \href{http://ucidatascienceinitiative.github.io/IntroToJulia/Html/JuliaMentalModel}{A
  Mental Model for Julia}
\item
  \href{http://ucidatascienceinitiative.github.io/IntroToJulia/Html/GithubIntroduction}{A
  Very Quick Introduction to Git/Github for Julia Users}
\end{itemize}

\subsubsection{Basics}\label{basics}

\begin{itemize}
\tightlist
\item
  \href{http://ucidatascienceinitiative.github.io/IntroToJulia/Html/BasicIntroduction}{A
  Basic Introduction to Julia}
\item
  \href{http://ucidatascienceinitiative.github.io/IntroToJulia/Html/WhyJulia}{Why
  Julia?}
\item
  \href{http://ucidatascienceinitiative.github.io/IntroToJulia/Html/ArraysAndMatrices}{More
  Details on Arrays and Matrices - How to get Fortran Speeds in Linear
  Algebra}
\item
  \href{https://docs.julialang.org/en/v1.0/}{The Julia Manual}
\item
  \href{https://en.wikibooks.org/wiki/Introducing_Julia}{The Julia
  Wikibook}
\item
  \href{https://docs.julialang.org/en/v1/manual/noteworthy-differences/}{Noteworthy
  Differences from Other Languages (Julia's Manual)}
\item
  \href{https://cheatsheets.quantecon.org/}{Julia Cheatsheet Reference
  (with MATLAB and Python)}
\end{itemize}

\subsubsection{General Problems}\label{general-problems}

\begin{itemize}
\tightlist
\item
  \href{http://ucidatascienceinitiative.github.io/IntroToJulia/Html/BasicProblems}{Basic
  Problems}
\item
  \href{http://ucidatascienceinitiative.github.io/IntroToJulia/Html/IntermediateProblems}{Intermediate
  Problems}
\item
  \href{http://ucidatascienceinitiative.github.io/IntroToJulia/Html/AdvancedProblems}{Advanced
  Problems}
\item
  \href{http://ucidatascienceinitiative.github.io/IntroToJulia/Html/StockProblem}{Stock
  Modeling Problem}
\end{itemize}

\subsubsection{Detailed Usage: A Peek Into "the Rabbit
Hole"}\label{detailed-usage-a-peek-into-the-rabbit-hole}

\begin{itemize}
\tightlist
\item
  \href{http://ucidatascienceinitiative.github.io/IntroToJulia/Html/DispatchDesigns}{Multiple
  Dispatch Designs: Duck Typing, Hierarchies and Traits}
\item
  \href{http://ucidatascienceinitiative.github.io/IntroToJulia/Html/Metaprogramming}{Metaprogramming}
\item
  \href{http://ucidatascienceinitiative.github.io/IntroToJulia/Html/CallOverloading}{Call
  Overloading}
\item
  \href{http://www.stochasticlifestyle.com/7-julia-gotchas-handle/}{7
  Julia Gotchas and How to Handle Them}
\item
  \href{http://ucidatascienceinitiative.github.io/IntroToJulia/Html/ArrayIteratorInterfaces}{Array
  and Iterator Interfaces}
\end{itemize}

\subsubsection{Packages to Explore (with
Problems!)}\label{packages-to-explore-with-problems}

This section introduces you to a wide variety of packages for data
science and scientific computing in Julia. Many of these pages have
example problems for you to have a guided tour through the package
basics.

\begin{itemize}
\tightlist
\item
  \href{http://ucidatascienceinitiative.github.io/IntroToJulia/Html/PackageEcosystem}{Overview
  of the Package Ecosystem}
\end{itemize}

\paragraph{Data Science}\label{data-science}

\begin{itemize}
\tightlist
\item
  \href{http://ucidatascienceinitiative.github.io/IntroToJulia/Html/Clustering}{Clustering:
  Clustering.jl and Distances.jl}
\item
  \href{http://ucidatascienceinitiative.github.io/IntroToJulia/Html/DimensionalityReduction}{Dimensionality
  Reduction: MultivariateStats.jl}
\item
  \href{http://ucidatascienceinitiative.github.io/IntroToJulia/Html/PlotsJL}{Data
  Visualization: Plots.jl}
\item
  Bioinformatics: Bio.jl
\item
  Deep Learning: Flux.jl
\end{itemize}

\paragraph{Scientific Computing}\label{scientific-computing}

\begin{itemize}
\tightlist
\item
  \href{http://ucidatascienceinitiative.github.io/IntroToJulia/Html/DiffEq}{Differential
  Equations: DifferentialEquations.jl}
\item
  \href{http://ucidatascienceinitiative.github.io/IntroToJulia/Html/NonlinearSolve}{Solving
  Nonlinear Equations: NLsolve.jl and Roots.jl}
\item
  \href{http://ucidatascienceinitiative.github.io/IntroToJulia/Html/Graphs}{Graph
  Algorithms and Analysis: LightGraphs.jl}
\end{itemize}

\paragraph{Both!}\label{both}

\begin{itemize}
\tightlist
\item
  \href{http://ucidatascienceinitiative.github.io/IntroToJulia/Html/Optimization}{Mathematical
  Programming / Optimization: JuMP and Optim.jl}
\item
  \href{http://ucidatascienceinitiative.github.io/IntroToJulia/Html/ForwardDiff}{Forward-mode
  Autodifferentiation (with Optimization): ForwardDiff.jl and Optim.jl}
\end{itemize}

\subsubsection{Extra Projects and
Problems}\label{extra-projects-and-problems}

\begin{itemize}
\tightlist
\item
  \href{http://ucidatascienceinitiative.github.io/IntroToJulia/Html/Interop}{Using
  External Languages from Julia (Interop)}
\item
  \href{http://ucidatascienceinitiative.github.io/IntroToJulia/Html/HPCJulia}{Parallelism
  and HPC}
\item
  \href{http://ucidatascienceinitiative.github.io/IntroToJulia/Html/PackageDevelopment}{Package
  Development, Documentation, and Testing}
\item
  Robust Benchmarking:
  \href{https://github.com/JuliaCI/BenchmarkTools.jl}{BenchmarkTools.jl}
  and \href{https://github.com/timholy/ProfileView.jl}{ProfileView.jl}
\end{itemize}

\subsubsection{Experiments}\label{experiments}

Probe around and understand the following results:

\begin{itemize}
\tightlist
\item
  \href{http://ucidatascienceinitiative.github.io/IntroToJulia/Html/TypeStabilityExperiment}{Type-Stability
  Experiment}
\item
  \href{http://ucidatascienceinitiative.github.io/IntroToJulia/Html/ScopingExperiment}{Scoping
  Experiment}
\end{itemize}

    \begin{longtable}[]{@{}l@{}}
\toprule
\# Problem Answers\tabularnewline
\bottomrule
\end{longtable}

\paragraph{Answers to the Problems}\label{answers-to-the-problems}

\begin{itemize}
\tightlist
\item
  \href{http://ucidatascienceinitiative.github.io/IntroToJulia/Html/BasicProblemAnswers}{Basic
  Problem Answers}
\item
  \href{http://ucidatascienceinitiative.github.io/IntroToJulia/Html/IntermediateProblemAnswers}{Intermediate
  Problem Answers}
\item
  \href{http://ucidatascienceinitiative.github.io/IntroToJulia/Html/AdvancedProblemAnswers}{Advanced
  Problem Answers}
\item
  \href{http://ucidatascienceinitiative.github.io/IntroToJulia/Html/StockProblemAnswers}{Answer
  to Stock Modeling Problem}
\item
  \href{http://ucidatascienceinitiative.github.io/IntroToJulia/Html/LightGraphsAnswers}{Answer
  to the LightGraphs Problem}
\item
  \href{http://ucidatascienceinitiative.github.io/IntroToJulia/Html/ForwardDiffAnswers}{Answer
  to the ForwardDiff Problems}
\item
  \href{http://ucidatascienceinitiative.github.io/IntroToJulia/Html/NonlinearSolveAnswers}{Answer
  to the Nonlinear Solver Problems}
\item
  \href{http://ucidatascienceinitiative.github.io/IntroToJulia/Html/DiffEqSolutions}{Answer
  to the DiffEq Problems}
\item
  \href{http://ucidatascienceinitiative.github.io/IntroToJulia/Html/OptimizationAnswers}{Answer
  to the Optimization Problems}
\item
  \href{http://ucidatascienceinitiative.github.io/IntroToJulia/Html/DimensionalityReductionSolutions}{Answer
  to Dimensionality Reduction Problem}
\item
  \href{http://ucidatascienceinitiative.github.io/IntroToJulia/Html/ClusteringSolutions}{Answer
  to Clustering Problem}
\end{itemize}

    \section{Slide Versions of the Pages}\label{slide-versions-of-the-pages}

\subsubsection{Introduction}\label{introduction}

\begin{itemize}
\tightlist
\item
  \href{http://ucidatascienceinitiative.github.io/IntroToJulia/Slides/ToolingDocumentationCommunity}{Tooling,
  Documentation, and Community}
\item
  \href{http://ucidatascienceinitiative.github.io/IntroToJulia/Slides/JuliaMentalModel}{A
  Mental Model for Julia}
\item
  \href{http://ucidatascienceinitiative.github.io/IntroToJulia/Slides/GithubIntroduction}{A
  Very Quick Introduction to Git/Github for Julia Users}
\end{itemize}

\subsubsection{Basics}\label{basics}

\begin{itemize}
\tightlist
\item
  \href{http://ucidatascienceinitiative.github.io/IntroToJulia/Slides/BasicIntroduction}{A
  Basic Introduction to Julia}
\item
  \href{http://ucidatascienceinitiative.github.io/IntroToJulia/Slides/BasicProblems}{Basic
  Problems}
\item
  \href{http://ucidatascienceinitiative.github.io/IntroToJulia/Slides/BasicProblems}{Intermediate
  Problems Problems}
\item
  \href{http://ucidatascienceinitiative.github.io/IntroToJulia/Slides/WhyJulia}{Why
  Julia?}
\end{itemize}

\subsubsection{Detailed Usage: A Peak Into "the Rabbit
Hole"}\label{detailed-usage-a-peak-into-the-rabbit-hole}

\begin{itemize}
\tightlist
\item
  \href{http://ucidatascienceinitiative.github.io/IntroToJulia/Slides/Metaprogramming}{Metaprogramming}
\item
  \href{http://ucidatascienceinitiative.github.io/IntroToJulia/Slides/CallOverloading}{Call
  Overloading}
\item
  \href{http://ucidatascienceinitiative.github.io/IntroToJulia/Slides/ArrayIteratorInterfaces}{Array
  and Iterator Interfaces}
\item
  \href{http://ucidatascienceinitiative.github.io/IntroToJulia/Slides/ArraysAndMatrices}{More
  Details on Arrays and Matrices - How to get Fortran Speeds from Linear
  Algebra}
\end{itemize}

\subsubsection{Packages to Explore}\label{packages-to-explore}

\begin{itemize}
\tightlist
\item
  \href{http://ucidatascienceinitiative.github.io/IntroToJulia/Slides/PackageEcosystem}{Overview
  of the Package Ecosystem}
\item
  \href{http://ucidatascienceinitiative.github.io/IntroToJulia/Slides/PlotsJL}{Data
  Visualization: Plots.jl}
\item
  \href{http://ucidatascienceinitiative.github.io/IntroToJulia/Slides/DiffEq}{Differential
  Equations: DifferentialEquations.jl}
\item
  \href{http://ucidatascienceinitiative.github.io/IntroToJulia/Html/Clustering}{Clustering:
  Clustering.jl and Distances.jl}
\item
  \href{http://ucidatascienceinitiative.github.io/IntroToJulia/Html/DimensionalityReduction}{Dimensionality
  Reduction: MultivariateStats.jl}
\item
  \href{http://ucidatascienceinitiative.github.io/IntroToJulia/Slides/Optimization}{Mathematical
  Programming / Optimization: JuMP and Optim.jl}
\item
  \href{http://ucidatascienceinitiative.github.io/IntroToJulia/Slides/NonlinearSolve}{Solving
  Nonlinear Equations: NLsolve.jl and Roots.jl}
\item
  \href{http://ucidatascienceinitiative.github.io/IntroToJulia/Slides/ForwardDiff}{Forward-mode
  Autodifferentiation (with Optimization): ForwardDiff.jl and Optim.jl}
\item
  \href{http://ucidatascienceinitiative.github.io/IntroToJulia/Slides/Graphs}{Graph
  Algorithms and Analysis: LightGraphs.jl}
\item
  Bioinformatics: Bio.jl
\item
  Deep Learning: KNet.jl, TensorFlow.jl, MXNet.jl, and Flux.jl
\end{itemize}

\subsubsection{Extra Projects and
Problems}\label{extra-projects-and-problems}

\begin{itemize}
\tightlist
\item
  \href{http://ucidatascienceinitiative.github.io/IntroToJulia/Slides/Interop}{Using
  External Languages from Julia (Interop)}
\item
  \href{http://ucidatascienceinitiative.github.io/IntroToJulia/Slides/HPCJulia}{Parallelism
  and HPC}
\item
  \href{http://ucidatascienceinitiative.github.io/IntroToJulia/Slides/JuliaML}{Work
  in Progress: JuliaML for Machine Learning}
\item
  \href{http://ucidatascienceinitiative.github.io/IntroToJulia/Slides/PackageDevelopment}{Package
  Development, Documentation, and Testing}
\item
  Tools for Faster Code:
  \href{https://github.com/simonbyrne/InplaceOps.jl}{InplaceOps.jl},
  \href{https://github.com/ahwillia/CatViews.jl}{CatViews.jl}, and
  \href{https://github.com/JuliaMath/VML.jl}{VML.jl}
\item
  Robust Benchmarking:
  \href{https://github.com/JuliaCI/BenchmarkTools.jl}{BenchmarkTools.jl}
  and \href{https://github.com/timholy/ProfileView.jl}{ProfileView.jl}
\end{itemize}

\subsubsection{Experiments}\label{experiments}

Probe around and understand the following results:

\begin{itemize}
\tightlist
\item
  \href{http://ucidatascienceinitiative.github.io/IntroToJulia/Slides/TypeStabilityExperiment}{Type-Stability
  Experiment}
\item
  \href{http://ucidatascienceinitiative.github.io/IntroToJulia/Slides/ScopingExperiment}{Scoping
  Experiment}
\end{itemize}

    \subsection{How These Were Made}\label{how-these-were-made}

This entire repository is made using Jupyter notebooks using the
template from the
\href{https://github.com/ChrisRackauckas/JupyterSite}{JupyterSite}
repository. To contribute to these materials, see
\href{https://github.com/UCIDataScienceInitiative/IntroToJulia}{the
Github repository}.


    % Add a bibliography block to the postdoc
    
    
    
    \end{document}
